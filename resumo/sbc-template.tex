\documentclass[12pt]{article}

\usepackage{sbc-template}
\usepackage{graphicx,url}
\usepackage[brazil]{babel}
\usepackage[utf8]{inputenc}
\usepackage{amsfonts}
\usepackage[cmex10]{amsmath}

\sloppy

\title{Transporte Cooperativo de Objetos utilizando Robôs Heterogêneos}

\author{Ramon Soares de Melo}

\address{
Instituto de Ciências Exatas (ICEx) - Universidade Federal de Minas Gerais (UFMG)
}

\begin{document}

\maketitle


\newcommand{\mrs}{MRS}
\newcommand{\environment}{\ensuremath{\mathbb{R}^3}}

% Shortcut definitions

\newcommand{\set}[1]{\ensuremath{\boldsymbol{\mathcal{#1}}}}
\newcommand{\setitem}[2]{\ensuremath{#1_{#2}}}
\newcommand{\setlist}[3]{#1\ $=$\ \{\setitem{#2}{1}, \setitem{#2}{2}, ..., \setitem{#2}{#3}\}}

\newcommand{\robotset}{\set{R}} % conjunto de robôs
\newcommand{\robotsetqt}{\ensuremath{k}}
\newcommand{\robot}[1]{\setitem{r}{#1}}
\newcommand{\robotlist}{\setlist{\robotset}{r}{\robotsetqt}}

\newcommand{\obstacleset}{\set{B}} % conjunto de obstáculos
\newcommand{\obstaclesetqt}{\ensuremath{x}}
\newcommand{\obstacle}[1]{\setitem{b}{#1}}
\newcommand{\obstaclelist}{\setlist{\obstacleset}{b}{\obstaclesetqt}}

\newcommand{\objectset}{\set{O}} % conjunto de objetos
\newcommand{\objectsetqt}{\ensuremath{y}}
\newcommand{\object}[1]{\setitem{o}{#1}}
\newcommand{\objectlist}{\setlist{\objectset}{o}{\objectsetqt}}

\newcommand{\planset}{\set{P}} % plano
\newcommand{\plansetqt}{\ensuremath{q}}
\newcommand{\plan}[1]{\setitem{n}{#1}}
\newcommand{\planlist}{\setlist{\planset}{n}{\plansetqt}}

\newcommand{\robotplanset}{\set{RP}} % conjunto de planos
\newcommand{\robotplansetqt}{\ensuremath{j}}
\newcommand{\robotplan}[1]{\setitem{rp}{#1}}
\newcommand{\robotplanlist}{\setlist{\robotplanset}{rp}{\robotplansetqt}}

\newcommand{\executionplanset}{\set{EP}} % conjunto de planos
\newcommand{\executionplansetqt}{\ensuremath{h}}
\newcommand{\executionplan}[1]{\setitem{ep}{#1}}
\newcommand{\executionplanlist}{\setlist{\executionplanset}{ep}{\executionplansetqt}}
\newcommand{\executionplanname}{\emph{Execution Plan}}
\newcommand{\executionplansetnotation}{\ensuremath{\executionplanset \subset \robotplanset}}

\newcommand{\executionplanminset}{\ensuremath{\executionplanset{*}}} % conjunto com custo minimo de segmentos

\newcommand{\segmentset}{\set{S}} % conjunto de segmentos
\newcommand{\segmentsetqt}{\ensuremath{e}}
\newcommand{\segment}[1]{\setitem{s}{#1}}
\newcommand{\segmentlist}{\setlist{\segmentset}{s}{\segmentsetqt}}

\newcommand{\segmentpointset}{\ensuremath{\set{S}_p}} % conjunto de pontos de segmentação
\newcommand{\segmentpointsetqt}{\ensuremath{z}}
\newcommand{\segmentpoint}[1]{\setitem{sp}{#1}}
\newcommand{\segmentpointfirst}[1]{\ensuremath{\segmentpoint{#1}^1}}
\newcommand{\segmentpointsecond}[1]{\ensuremath{\segmentpoint{#1}^2}}

\newcommand{\segmentpointlist}{\setlist{\segmentpointset}{sp}{\segmentpointsetqt}}

\newcommand{\typeland}{ground}
\newcommand{\typeaerial}{aerial}
\newcommand{\typeset}{\set{T}} % conjunto de tipos de movimentação
\newcommand{\typelist}{\ensuremath{\typeset=\ \{}\typeland, \typeaerial\ensuremath{\}}}
\newcommand{\type}[1]{\setitem{t}{#1}}

\newcommand{\plantypestart}{initial}
\newcommand{\plantypetransition}{transition}
\newcommand{\plantypemove}{movement}
\newcommand{\plantypeset}{\set{TP}} % conjunto de tipos de plano
\newcommand{\plantypelist}{\ensuremath{\plantypeset=\ \{}\plantypestart, \plantypetransition, \plantypemove\ensuremath{\}}}
\newcommand{\plantype}[1]{\setitem{tp}{#1}}

\newcommand{\movementtypepremove}{pre-transport}
\newcommand{\movementtypemove}{transport}
\newcommand{\movementtypeset}{\set{T_m}} % conjunto de tipos de plano
\newcommand{\movementtypelist}{\ensuremath{\movementtypeset=\ \{}\movementtypepremove, \movementtypemove\ensuremath{\}}}
\newcommand{\movementtype}[1]{\setitem{tm}{#1}}

\newcommand{\tokenset}{\set{TO}}
\newcommand{\tokensetqt}{\ensuremath{u}}
\newcommand{\tokeni}[1]{\setitem{to}{#1}}
\newcommand{\token}{\emph{token}}
\newcommand{\tokenlist}{\setlist{\tokenset}{to}{\tokensetqt}}

\newcommand{\workspace2}{\ensuremath{\boldsymbol{\mathcal{W}}}} % àrea de trabalho
\newcommand{\workspacecell}{\ensuremath{\boldsymbol{\mathcal{c}}}} % àrea de trabalho

\newcommand{\allocationgraph}{\ensuremath{\mathcal{AG}}}
\newcommand{\allocationgraphcompress}{\ensuremath{\mathcal{AG}_c}}

\newcommand{\currentstate}{\ensuremath{S}}
\newcommand{\nextstate}{\ensuremath{S'}}
\newcommand{\originstate}{\ensuremath{S_o}}
\newcommand{\targetstate}{\ensuremath{S_d}}
\newcommand{\robotstate}{\ensuremath{S_r}}

\newcommand{\robotinitialstate}{\ensuremath{I}}
\newcommand{\robotinitialstatei}[1]{\ensuremath{I_{#1}}}

\newcommand{\celldimension}{\ensuremath{d}}
\newcommand{\deslocationfactor}{\ensuremath{l}}

\newcommand{\movementset}{\set{M}}
\newcommand{\movementslist}{\ensuremath{\movementset=\ \{}left, right, front, back, up, down\ensuremath{\}}}
\newcommand{\movementaction}{\ensuremath{a_i}}

% Funções

\newcommand{\utilityfunction}{\ensuremath{\Theta}}
\newcommand{\utilityplanfunction}{\ensuremath{\utilityfunction_{p}}}
\newcommand{\utilitytotalfunction}{\ensuremath{\utilityfunction_{t}}}
\newcommand{\distancefunction}{\ensuremath{\Delta}}
\newcommand{\timefunction}{\ensuremath{\Upsilon}}
\newcommand{\energyfunction}{\ensuremath{\Psi}}

% Planejamento

\newcommand{\fringe}{\set{F}}
\newcommand{\searchednodes}{\set{SN}}
\newcommand{\node}{\ensuremath{n}}
\newcommand{\nodeitem}[1]{\ensuremath{\node_{#1}}}
\newcommand{\nodeparent}{\emph{NodePai}}
\newcommand{\nodeutility}{\ensuremath{\omega}}
\newcommand{\nodedata}{\ensuremath{\{}state (\currentstate), action (\movementaction), utility (\nodeutility), agent's position (\robotstate), type (\type{i})\ensuremath{\}}}


% A utilização de sistemas multi-robô em diversos contextos e aplicações diferentes tem aumentado significativamente nos últimos anos. Um dos principais desafios é o transporte de objetos, que pode ser aplicado em casos simples, como mover objetos em um ambiente bem como em cenários mais complexos, tais como tarefas de construção e criação de estruturas. Apesar do fato de que muito esforço tem sido focado no que pode aparentemente ser uma tarefa relativamente simples, várias facetas do problema ainda permanecem em aberto e precisam ser enfrentadas. Neste trabalho, propomos uma metodologia completa que abrange todas as etapas relacionadas do problema: planejamento de trajetória, alocação de tarefas e coordenação.
% Numerosos ensaios para experimentos com um terreno robôs aéreos, e em solo com robôs reais foram conduzidos a fim de fornecer uma avaliação completa e validação da metodologia.

% \small
\footnotesize

% Motivação

A aplicação de sistemas autônomos pode ser observada em diversos cenários, como atividades de vigilância, busca e resgate, rastreamento, mapeamento, dentre estes, o transporte e manipulação de objetos é uma das quais tem sido mais exploradas, principalmente por acrescentarem a atividade características como precisão, pontualidade, agilidade, controle, além de poder garantir uma melhor eficiência e eficácia.

Apesar de robôs em alguns casos ainda não demonstrarem a mesma destreza dos humanos, sua utilização é mais indicada para ambiente hostís, como locais com fogo e/ou fumaça, grandes profundidades no oceano e áreas com contaminação nuclear, por exemplo.
Além destes casos, também é possível observar seu uso em um ambiente residencial, onde podem tratar de tarefas domésticas, como controlar medicamentos ou a limpeza, além da assistência a idosos ou paciente com dificuldade de locomoção.

% \section{Problema} % (fold)

% esta problemática pode ser descrita como:

Este trabalho tem por objetivo demonstrar uma técnica que apresente soluções para questões decorrentes do uso de uma equipe de agentes robóticos, tendo como missão o transporte de objetos utilizando de suas capacidades físicas e computacionais.
Formalmente, dado um ambiente definido em um espaço euclidiano \environment, denominado $\workspace \subset \environment$, como área de trabalho, temos neste, os conjuntos (i) \objectset\ descrevendo todos os objetos que devem ser transportados, (ii) \obstacleset\ com os obstáculos presentes no ambiente, e (iii) \robotset\ o conjunto de agentes responsáveis por realizar a tarefa de transporte.

% Metodologia

O processo de transporte de objetos envolve diversas etapas que são estudadas e tratadas neste trabalho, são elas: (i) modelagem do ambiente de trabalho, (ii) planejamento (iii) alocação de tarafas e (iv) coordenação e executação.
Diferente de outros trabalhos que descrevem uma solução para a manipulação de objetos, focados principalmente na trajetória executada pelos agentes, neste, o plano de movimentação do objeto é utilizado como guia para as demais fases.
Além disto, uma função de utilidade (\utilityfunction) que engloba dimensões como tempo de deslocamento, energia gasta e distância até o destino, é utilizada para mensurar a qualidade dos planos criados, e assim assegurar que a melhor estratégia seja executada.

Neste sentido, baseado nas capacidades de transporte dos agentes (empurrar, elevar, etc) do conjunto \robotset, planos de transporte são criados para os objetos do conjunto \objectset, considerando a função de utilidade \utilityfunction e o ambiente \workspace.
Munido destes planos de movimentação, são criados planos similares, porém para os agentes, podendo classificados em dois tipos: (i) preparação - no qual o agente se aproxima do objeto a ser transportado, (ii) transporte - onde o objeto é transportado.

Os planos gerados para os agentes podem possuir valores de utilidade diferentes, provenientes das diferentes capacidades dos mesmos, como o tipo de transporte realizado, ou a distância que o mesmo se encontra do objeto, por exemplo.
A fim de selecionar os melhores planos para realizar o transporte e alocar dentre os agentes os trajetos a serem realizados, estes são organizados em forma de um grafo direcionado, criando uma interligação dentre todos os planos.
Neste grafo é aplicado o algoritmo de \emph{Kruskal}, de modo a gerar um grafo com somente as arestas (planos) de melhor utilidade, tendo por consequência, a alocação de tarefas entre os agentes mais aptos para executar o transporte.

A coordenação dos agentes para realizaçãos de seus respectivos planos acontece mediante a troca de \emph{tokens} entre os mesmos. Um \emph{token} representa uma sinalização que um determinado tipo (preparação, transporte) de plano pode ser executado.
Deste modo, é determinado que podem existir vários \emph{tokens} para o tipo preparação, porém, somente um (1) do tipo transporte, assegurando que agentes disponíveis cumpram as fases de preparação, e o agente responsável pelo transporte o execute sem que outros interfiram no mesmo.
Este processo é repetido até que todos os objetos sejam transportados.

Mediante estes passos, é possivel realizar o transporte de objetos, considerando as capacidades de um conjunto heterogêneo de agentes visando maximizar a utilidade do sistema ponderando dimensões que tem direta influênica na qualidade da tarefa desempenhada.

\nocite{Ahmadabadi2001}
\nocite{Miyata2002}
\nocite{Gerkey2002}
\nocite{Sugar2002}
\nocite{Song2002}
\nocite{Tanner2003}
\nocite{Yamashita2003}
\nocite{Costa2007}
\nocite{Zacharias2007}
\nocite{Shiroma2009}
\nocite{Ozturk2014}
\nocite{Carvalho2013}
\nocite{Fink2008}
\nocite{Gerkey2001}
\nocite{Kim2013}
\nocite{Michael2011}
\nocite{Sujit2013}
\nocite{Tiganas2013}
\nocite{Wawerla2010}

\newpage

\bibliographystyle{sbc}
\bibliography{sbc-template}

\end{document}
