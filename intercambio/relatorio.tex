- Introdução

A utilização de veículos não tripulados vem crescendo conforme a disponibilidade deste tipo de equipamento se torna cada vez maior. Os mesmo podem ser utilizados para diversas atividade nas quais um piloto a bordo não é estritamente necessário, bem como podem ser realizadas missões totalmente automáticas, sem necessidade de um piloto.

Estes equipamentos são principalmente classificados quanto ao seu tipo de locomoção, seja ele terrestre, aéreo ou aquático, respectivamente como (siglas dos nomes). Cada tipo com características especificas, como autonomia de funcionamento ou capacidade de payload.

Uma das plataforma de controle e processamento de dados destes tipos de veículos é o chamado ArduPilot ou APM, que é uma unidade de processamento que agrega uma serie de sensores além de portas de controle, sendo capaz de atuar no nível mais baixo de controle do veículo para que execute ações predeterminadas.

Afim de realizar a conversação entre este tipo de autopilotos, é utilizada o protocolo de comunicação MAVLink, que descreve  extenso vocabulários de comandos que podem ser utilizados para trocar informações entre o controlador e uma estação base, ou mesmo entre auto pilotos.

Por meio deste conjunto, um veículo, um auto piloto, e o protocolo de comunicado, atividades como a varredura e captura de imagens de uma determinada área, realização de missões autônomas dentre outras, podem ser realizadas de forma consistente e facilmente configuráveis.

- Metodologia
- Resultados
