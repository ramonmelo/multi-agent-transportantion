The application of autonomous systems can be observed in various scenarios, such as surveillance, search and rescue, tracking and mapping, apart from these, transportation and manipulation of objects is a kind of mission that has gained attention from researchers and large companies, mainly by adding to the activity features like precision, speed and control, in addition to ensuring overall efficiency improvements to the system.

This work demonstrated a set of techniques used to coordinate a team of heterogeneous robotic agents, whose mission is to transport objects using their physical and computational skills.
The process of transport and manipulate is handled in three stages: (i) object path planning, in which a plan is created based on metrics of energy use and cost of time; (ii) planning and task allocation between the available agents for transportation, making the distribution of tasks in order to minimize the total utility of the system; (iii) coordination and execution, in which agents perform transport synchronously.

Were performed quantitative experiments, demonstrating the effectiveness of the method valued in several different scenarios, as well as simulated and real tests, illustrating the manipulation of objects by robotic agents themselves.
Thus, it is shown a methodology capable of performing a full transport mission, improving it through the use of intelligent agents.

\keywords{Cooperative Transport, Object Manipulation, Task Allocation}
