u% \iffalse meta-comment
%
% Copyright (C) 2004,2005,2006,2007,2008,2009,2010,2011 by Vilar Fiuza da
% Camara Neto and Eduardo Freire Nakamura.
%
% This file may be distributed and/or modified under the
% conditions of the LaTeX Project Public License, either
% version 1.2 of this license or (at your option) any later
% version.  The latest version of this license is in:
%
%     http://www.latex-project.org/lppl.txt
%
% and version 1.2 or later is part of all distributions of
% LaTeX version 1999/12/01 or later.
%
% Currently this work has the LPPL maintenance status "maintained".
%
% The Current Maintainer of this work is Vilar Fiuza da Camara Neto.
%
% This work consists of the files ppgccufmg.dtx and
% ppgccufmg.ins and the derived file ppgccufmg.cls.
%
% \fi
%
% \iffalse
%<*driver>
\ProvidesFile{ppgccufmg.dtx}
%</driver>
%<class>\NeedsTeXFormat{LaTeX2e}[1999/12/01]
%<class>\ProvidesClass{ppgccufmg}
%<*class>
  [2012/09/17 v1.51 Classe para dissertacoes, teses e propostas de tese no
  formato PPGCC/ICEx/UFMG]
%</class>
%<*driver>
\documentclass{ltxdoc}
\usepackage[latin1]{inputenc}
\usepackage[T1]{fontenc}
\usepackage[brazil]{babel}

\EnableCrossrefs
\CodelineIndex
\RecordChanges

\begin{document}
\DocInput{ppgccufmg.dtx}
\end{document}
%</driver>
% \fi

% \CheckSum{2223}


% \CharacterTable
%  {Upper-case    \A\B\C\D\E\F\G\H\I\J\K\L\M\N\O\P\Q\R\S\T\U\V\W\X\Y\Z
%   Lower-case    \a\b\c\d\e\f\g\h\i\j\k\l\m\n\o\p\q\r\s\t\u\v\w\x\y\z
%   Digits        \0\1\2\3\4\5\6\7\8\9
%   Exclamation   \!     Double quote  \"     Hash (number) \#
%   Dollar        \$     Percent       \%     Ampersand     \&
%   Acute accent  \'     Left paren    \(     Right paren   \)
%   Asterisk      \*     Plus          \+     Comma         \,
%   Minus         \-     Point         \.     Solidus       \/
%   Colon         \:     Semicolon     \;     Less than     \<
%   Equals        \=     Greater than  \>     Question mark \?
%   Commercial at \@     Left bracket  \[     Backslash     \\
%   Right bracket \]     Circumflex    \^     Underscore    \_
%   Grave accent  \`     Left brace    \{     Vertical bar  \|
%   Right brace   \}     Tilde         \~}
%
% \changes{v1.51}{2012-09-17}{Ajustes na Ficha Catalogr?ica}
% \changes{v1.50}{2012-09-12}{Inclus? das chaves `isbn' e `issn' no estilo do
%   BibTeX; suporte a informa?es sobre coorienta?o; suporte a v?ias p?inas
%   de Folha de Aprova?o}
% \changes{v1.44}{2011-04-25}{Corre?o de bugs no formato BibTeX;
%   flexibiliza?o da formata?o de valores num?icos; configura?o manual das
%   chaves para indexa?o da Ficha Catalogr?ica; corre?o da Folha de Aprova?o}
% \changes{v1.43}{2011-01-31}{Tradu?o de alguns termos do estilo do BibTeX;
%   corre?o de p?ina em branco no fim do documento e de contagem do n?mero de
%   p?inas do frontmatter; corre?o de bugs}
% \changes{v1.42}{2010-04-22}{Corre?o no timing de interpreta?o da data do
%   documento}
% \changes{v1.41}{2009-06-30}{Corre?o de timing para a constru?o correta do
%   Sum?io, no que diz respeito ao t?ulo de ap?dices e anexos}
% \changes{v1.40}{2009-04-22}{Altera?o da posi?o do Sum?io; op?o para a
%   inser?o de material antes do Sum?io}
% \changes{v1.31}{2009-04-07}{Retirada de r?ua do cabe?lho no modo `oneside'}
% \changes{v1.30}{2009-03-12}{Ajuste de alguns detalhes de formata?o}
% \changes{v1.21}{2008-10-20}{\string\cleardoublepage antes de
%   \string\tableof...}
% \changes{v1.20}{2008-10-16}{Cria?o do comando \string\keywords; op?o
%   `abstract' permite especificar a linguagem do resumo; descri?o do
%   documento inclui ``Instituto de Ci?cias Exatas''; mudan? na posi?o do
%   nome do orientador na Folha de Rosto}
% \changes{v1.10}{2008-10-02}{Ficha Catalogr?ica condicional (n? ?criada
%   para projetos e propostas); comando \string\sloppy ?disparado
%   automaticamente; o pacote n? comp?e mais a Folha de Aprova?o; mudan? na
%   contagem das p?inas introdut?ias; inclus? da p?ina de ep?rafe; estilo
%   de bibliografia alterado de ufmg.bst para ppgccufmg.bst}
% \changes{v1.0}{2008-05-09}{Nome da classe alterada de `ufmgthesis' para
%   `ppgccufmg'; mudan? na contagem das p?inas introdut?ias; inclus? da
%   Ficha Catalogr?ica}
% \changes{v0.31}{2008-02-26}{Corre?o da contra-capa, em que aparecia o nome
%   do autor duplicado}
% \changes{v0.30}{2007-12-13}{Mudan? na especifica?o de dedicat?ias e
%   agradecimentos; cabe?lho da p?ina de t?ulo agora apresenta nome do autor
%   e do orientador}
% \changes{v0.20}{2007-11-09}{Mudan? na especifica?o de resumos e
%   \emph{abstracts}; t?mino da tradu?o da documenta?o para o portugu?}
% \changes{v0.14}{2006-11-05}{Documenta?o traduzida para o portugu?.
%   Eliminados brokentitle e similares. Corrigidos bugs de formata?o.
%   Corrigido cabe?lho quando ap?dices s? postos ap? as Refer?cias
%   Bibliogr?icas}
% \changes{v0.13}{2005-07-24}{The approval sheet now shows the broken title}
% \changes{v0.12}{2005-07-21}{Bugs: arabic numbering in front matter (with
%   ``easy way'') and no header in main matter (with command
%   \string\ufmgthesis)}
% \changes{v0.11}{2005-06-21}{Typo: \string\ufmgthesis was previously named
%   \string\ufmghesis; Front matter header/footer reformat; page numbering
%   starts right after cover/title pages; \string\mainmatter is now automatic
%   (see `nomainmatter' class option)}
% \changes{v0.10}{2005-03-12}{Cover/title page reformat; added `centertitles'
%   class option}
% \changes{v0.09}{2005-03-10}{Extended the Portuguese/non-Portuguese
%   integration}
% \changes{v0.08}{2005-03-09}{Changed date format and document structure,
%   especially to provide more integration between Portuguese and
%   non-Portuguese parts}
% \changes{v0.07}{2005-03-09}{Changed ``Thesis proposal'' to ``Thesis
%   project''}
% \changes{v0.06}{2005-03-07}{Corrected the bibliography header placement}
% \changes{v0.05}{2005-02-22}{Added bibliography support}
% \changes{v0.04}{2005-01-15}{Added `memhfixc' package auto-loading}
% \changes{v0.03}{2005-01-15}{`twoside' option is working now}
% \changes{v0.02}{2005-01-13}{Corrected small formatting bugs}
% \changes{v0.01}{2004-12-09}{Initial version (first draft)}
%
% \GetFileInfo{ppgccufmg.dtx}
%
% \newcommand\cls{\textsf}
% \newcommand\pkg{\textsf}
% \newcommand\bst{\textsf}
%
% \title{A classe \cls{ppgccufmg}\thanks{Este documento corresponde ?vers?
% \fileversion{} da classe \cls{ppgccufmg}, lan?da em~\filedate.}}
% \author{Vilar Fiuza da Camara Neto\qquad\qquad Eduardo Freire Nakamura \\
% \texttt{\{neto,nakamura\}@dcc.ufmg.br}} \date{12 de setembro de 2012}
%
% \sloppy
% \maketitle
%
% \StopEventually
%
% \tableofcontents
%
%
% \section{Introdu?o}
%
% Este arquivo descreve a classe \LaTeX{} \cls{ppgccufmg} --- uma classe para
% formatar disserta?es de mestrado, teses de doutorado e propostas de tese
% para o Programa de P?-Gradua?o em Ci?cia da Computa?o da Universidade
% Federal de Minas Gerais (PPGCC/UFMG). Durante um bom tempo esta classe se
% chamou \pkg{ufmgthesis} e foi utilizada por v?ios alunos em car?er
% extra-oficial, embora incorporando sugest?es tanto da Secretaria quanto da
% Coordena?o do Curso. Por sua vez, a \pkg{ufmgthesis} ?baseada na classe
% original \pkg{ufmgthes} criada pelo estudante Eduardo Freire Nakamura.
%
% A vers? atual da classe est?em conformidade com a padroniza?o oficial
% estabelecida pelo PPGCC/UFMG.
%
%
% \section{Como usar a classe}
%
% Esta se?o descreve como usar a classe \cls{ppgccufmg}.
%
% As subse?es mais importantes s? a \ref{sec:classoptions} (op?es da classe,
% ou seja, as op?es para o comando |\documentclass|) e a \ref{sec:ppgccufmg}
% (como usar o comando |\ppgccufmg|, que monta o documento automaticamente).
% Para usu?ios de vers?es anteriores desta classe (ou da classe
% \pkg{ufmgthesis}), cheque tamb? a subse?o \ref{sec:pastusers}, para checar
% se houve alguma mudan? na sintaxe dos comandos ou na forma de prover as
% informa?es do documento.
%
% Note que a classe \cls{ppgccufmg} ?baseada na classe \cls{memoir}, que ?uma
% classe altamente configur?el. Voc?deve t?la instalada no seu sistema
% \LaTeX. Alguns sistemas j?v? com essa classe (incluindo a distribui?o mais
% comum do Windows, o MiK\TeX). A Subse?o~\ref{sec:memoirtips} cont? uma
% descri?o r?ida sobre a instala?o do \cls{memoir}.
%
%
% \subsection{Para os impacientes}
% \label{sec:quickintro}
%
% Esta ?a estrutura de um arquivo \LaTeX{} baseado no estilo \cls{ppgccufmg},
% j?com alguns pacotes recomendados:
%
% \begin{quote}
% \begin{verbatim}
% \documentclass[phd]{ppgccufmg}    % ou [msc] para disserta?es
%                                   % de mestrado. Para propostas ou
%                                   % projetos, usar [phd,project],
%                                   % [msc,proposal], etc.
%
% \usepackage[brazil]{babel}        % ajusta v?ios detalhes para
%                                   % documentos escritos em portugu?,
%                                   % principalmente hifeniza?o
% \usepackage[T1]{fontenc}          % permite a hifeniza?o de
%                                   % palavras acentuadas
% \usepackage[latin1]{inputenc}     % ou [utf8x] para quem prefere
%                                   % usar a codifica?o UTF-8
% \usepackage{graphicx}             % define o comando \includegraphics
%                                   % para a inclus? de figuras
% \usepackage[square]{natbib}       % permite cita?es naturalmente
%                                   % integradas ao texto
%
% \begin{document}
%
% \ppgccufmg{
%   % inserir aqui as informa?es para o comando \ppgccufmg
% }
%
% \chapter{Introdu?o}
%
% % a partir daqui vem o texto do seu documento
%
% % bibliografia:
% \ppgccbibliography{nome do arquivo BibTeX, sem a extens?}
%
% % ap?dices, se houver
% \begin{appendices}
%
% \chapter{Um ap?dice}
% ...conte?do do ap?dice...
%
% \chapter{Outro ap?dice}
% ...conte?do do ap?dice...
%
% \end{appendices}
%
% % anexos, se houver
% \begin{attachments}
%
% \chapter{Um anexo}
% ...conte?do do anexo...
%
% \chapter{Outro anexo}
% ...conte?do do anexo...
%
% \end{attachments}
%
% \end{document}
%\end{verbatim}
% \end{quote}
%
% A descri?o do comando |\ppgccufmg| ?vista na Subse?o~\ref{sec:ppgccufmg}.
%
% Para quem quiser usar outras listas (Lista de Algoritmos, de S?bolos, etc.),
% siga os seguintes passos:
%
% \begin{enumerate}
%
% \item \emph{N? altere o arquivo} |ppgccufmg.cls|;
%
% \item Use a op?o |beforetoc| do comando |\ppgccufmg| para incluir os
% comandos que devem ser executados antes do Sum?io:
% \begin{quote}
% \begin{verbatim}
% \ppgccufmg{...,beforetoc={\listofalgorithms}}
%\end{verbatim}
% \end{quote}
%
% Todos os comandos indicados como par?etro para a op?o |beforetoc| s?
% executados imediatamente antes da cria?o do Sum?io. Se a seq??cia de
% comandos for mais extensa (por exemplo, se voc?comp?e manualmente uma tabela
% para a Lista de S?bolos), ent? o melhor ?colocar esses comandos em um
% arquivo separado --- digamos, |listadesimbolos.tex| --- e usar a op?o
% |beforetoc| da seguinte maneira:
% \begin{quote}
% \begin{verbatim}
% \ppgccufmg{...,beforetoc={\input{listadesimbolos.tex}}}
%\end{verbatim}
% \end{quote}
%
% \end{enumerate}
%
%
% \subsection{Informa?es importantes para usu?ios de vers?es anteriores}
% \label{sec:pastusers}
%
% Se voc?estava usando uma vers? anterior da classe \cls{ppgccufmg} e est?% atualizando-a para a vers? atual, algumas mudan?s estruturais (por exemplo,
% na sintaxe dos comandos) podem ter ocorrido. D?uma olhada nas subse?es a
% seguir para ver se essas mudan?s afetam o seu documento.
%
% Naturalmente, se voc?est?usando o \cls{ppgccufmg} pela primeira vez,
% esta se?o n? ?importante.
%
% \subsubsection{Mudan?s introduzidas na vers? 1.40}
%
% O comando |\ppgccufmg| possui uma nova op?o: |beforetoc|. Esse comando deve
% ser usado para listar os comandos que s? executados antes do Sum?io, em
% particular para a gera?o de listas extras, como a Lista de Algoritmos ou de
% S?bolos.
%
% \subsubsection{Mudan?s introduzidas na vers? 1.20}
%
% O comando \cmd{\keywords} foi criado para automatizar a especifica?o das
% palavras-chave em resumos e \emph{abstracts}. O antigo comando
% \cmd{\keywords} foi alterado para \cmd{\fckeywords}, para separar a
% funcionalidade espec?ica da Ficha Catalogr?ica.
%
% A op?o |abstract| permite especificar a linguagem do resumo em quest?.
% Recomenda-se usar |abstract=[english]{Abstract}{arquivo}| para o
% \emph{abstract} (resumo em ingl?) --- assim o comando \cmd{\keywords} gera o
% termo em ingl? ``Keywords''.
%
% Foram criados os ambientes |appendices| e |attachments| para facilitar a
% cria?o de ap?dices e anexos.
%
% \subsubsection{Mudan?s introduzidas na vers? 1.10}
%
% O comando |\ufmgbibliography| foi alterado para |\ppgccbibliography|, para
% refletir o fato de que o estilo n? ?adotado pela UFMG como um todo (somente
% pelo PPGCC).
%
% O padr? para o tamanho da fonte foi mudado para 12pt, ao inv? de 11pt.
%
% Como a Folha de Aprova?o ?um documento fornecido pela Secretaria do Curso,
% n? faz sentido que esta classe gere essa p?ina. Assim, as seguintes
% mudan?s foram feitas:
%
% \begin{itemize}
% \item A sintaxe da op?o |advisor| mudou. Agora ?necess?io informar somente
%   o nome do orientador, n? mais o seu t?ulo e institui?o. A mesma mudan?
%   se aplica ao comando |\advisor|;
% \item Todas as op?es de |\ppgccufmg| e comandos para especificar as
%   informa?es da Folha de Aprova?o foram suprimidas. Assim, as op?es
%   |coadvisor|, |member| e |logo| n? s? mais reconhecidas, assim como os
%   comandos |\coadvisor|, |\addtocomitee| e |\logo|;
% \item A Folha de Aprova?o deve ser digitalizada ap? a emiss? pela
%   Secretaria (preferencialmente em formato |png| se voc?estiver usando o
%   |pdflatex| ou em |eps| se estiver usando o |latex|). O nome do arquivo
%   correspondente deve ser informado pela op?o |approval| (ou pelo comando
%   |\approval|).
% \end{itemize}
%
% Fora isso, a numera?o das p?inas mudou novamente: a falsa folha de rosto
% (antes chamada de ``capa'') n? conta na numera?o.
%
% \subsubsection{Mudan?s introduzidas na vers? 1.0}
%
% Esta ?a primeira vers? oficial da classe. As principais mudan?s s? as
% seguintes:
%
% \begin{itemize}
% \item o nome da classe foi mudado de \pkg{ufmgthesis} para
%   \cls{ppgccufmg} (e o comando |\ufmgthesis| tamb? foi mudado para
%   |\ppgccufmg|);
% \item recomanda-se usar a chave |authorrev| ao inv? de |author|;
% \item as op?es de classe |a4paper,11pt,twoside| s? adotadas por padr?
%   (n? ?necess?io cit?las no |\documentclass|);
% \item foi criada a Ficha Catalogr?ica;
% \item foi alterada a numera?o das p?inas introdut?ias (a p?ina `i' ?a
%   capa).
% \end{itemize}
%
% \subsubsection{Mudan?s introduzidas na vers? 0.30}
%
% Por sugest? da Secretaria do nosso curso, decidimos (e achei melhor) colocar
% a Dedicat?ia e os Agradecimentos antes dos resumos. Al? disso, mudou o
% cabe?lho da p?ina de t?ulo: ao inv? do nome da institui?o, passa a
% aparecer os nomes do autor e o orientador.
%
% \subsubsection{Mudan?s introduzidas na vers? 0.20}
%
% A defini?o de resumos e \emph{abstracts} mudou: os diversos comandos
% |\abstract|, |\englishabstract| e |\portugueseabstract| foram centralizados
% em um ?nico comando |\abstract|. A sintaxe deste comando ?% |\abstract|\marg{t?ulo}\marg{arquivo} (note que especificar o t?ulo ?% obrigat?io, como ``Resumo'', ``Abstract'' ou ``Resumo Estendido''). ?% poss?el definir qualquer n?mero de resumos e \emph{abstracts}.
%
% A mesma observa?o se aplica ? op?es |abstract|, |englishabstract| e
% |portugueseabstract| do comando |\ufmgthesis|, substitu?as pela op?o
% |abstract|.
%
% \subsubsection{Mudan?s introduzidas na vers? 0.14}
%
% As op?es |brokentitle| e |portuguesebrokentitle| e os comandos
% |\brokentitle| e |\portuguesebrokentitle| foram eliminados. Sem d?nem
% piedade. Caso voc?queira definir manualmente as quebras de linha para
% balancear melhor o t?ulo na capa e na folha de t?ulo, introduza as quebras
% de linha (|\\|) na pr?ria op?o |title| ou no comando |\title|.
%
% O comando |\backmatter| n? ?mais chamado automaticamente. Isso resolve
% o problema da numera?o inexistente dos ap?dices, para quem prefere
% coloc?los ap? as Refer?cias Bibliogr?icas. (Entretanto, alguns recomendam
% que as Refer?cias Bibliogr?icas sejam colocadas \emph{ap?} os ap?dices.)
%
% \subsubsection{Mudan?s introduzidas na vers? 0.11}
%
% \textbf{Importante!} Vers?es anteriores tinham um erro de grafia no comando
% |\ufmgthesis|: seu nome era definido como |\ufmghesis| (ou seja, sem o
% \textbf{t}). Esse erro foi corrigido a partir da vers? 0.11. Se voc?estiver
% atualizando de uma vers? anterior, por favor altere o seu documento para
% chamar o comando com a grafia correta.
%
% O comando |\mainmatter| agora ?chamado automaticamente, a n? ser que a
% op?o de classe |nomainmatter| seja especificada.
%
% A formata?o dos cabe?lhos e rodap? mudou: o t?ulo da se?o n? aparece
% mais, porque muitas vezes o cabe?lho ficava sobrecarregado com muito texto
% (especialmente no caso de se?es com t?ulos compridos).
%
% Al? disso, a numera?o das p?inas come? logo ap? a folha de t?ulo. (Nas
% vers?es anteriores, a numera?o come?va somente ap? o Sum?io.)
%
% \subsubsection{Mudan?s introduzidas na vers? 0.10}
%
% A formata?o da capa e da p?ina de t?ulo mudou. Na nova vers?, a posi?o
% do nome do autor do documento ?diferente.
%
% \subsubsection{Mudan?s introduzidas na vers? 0.09}
%
% Foram criadas as op?es |portuguesetitle|, |portuguesebrokentitle|,
% |portugueseuniversity| e |portuguesecourse|. Elas s?s? importantes se voc?% estiver escrevendo o documento em uma l?gua diferente do portugu?. Al?
% disso, a op?o |department| desapareceu, pois essa informa?o n? era usada.
%
% \subsubsection{Mudan?s introduzidas na vers? 0.08}
%
% O formato da data fornecido pela op?o |date=|\meta{data} (para o comando
% |\ufmgthesis|) ou pelo comando |\date|\marg{data} mudou. Ao inv? de fornecer
% a data de forma textual (ou seja, ``10 de janeiro de 2005'' ou ``January 10,
% 2005''), voc?deve agora fornecer a data no formato |aaaa-mm-dd| ou
% |aaaa-mm|. Veja a Se?o~\ref{sec:ppgccufmg} para mais detalhes.
%
%
% \subsection{Op?es da classe}
% \label{sec:classoptions}
%
% \newcommand{\defmark}{$\triangleright$}
% \newcommand{\sep}{$\mid$}
%
% Ao usar o comando |\documentclass|, voc?pode fornecer algumas op?es para
% configurar a apar?cia do documento.
%
% \emph{Para os impacientes:} o mais comum ?usar somente o seguinte (para
% teses de doutorado):
% \begin{quote}
% |\documentclass[phd]{ppgccufmg}|
% \end{quote}
% Para disserta?es de mestrado, basta colocar |msc| no lugar de |phd|. No caso
% de documento de proposta ou projeto, usar tamb? a op?o |proposal| ou
% |project|, ou seja:
% \begin{quote}
% |\documentclass[phd,project]{ppgccufmg}|
% \end{quote}
% para o projeto de tese.
%
% As op?es dispon?eis s? listadas a seguir. A op?o padr? em cada caso ?% identificada pelo s?bolo \defmark.
%
% \begin{itemize}
%
% \item |msc| \sep{} \defmark|phd|
%
%   Define se este documento ?uma disserta?o de mestrado ou uma tese de
%   doutorado.
%
% \item |proposal| ou |project|
%
%   Se qualquer uma dessas op?es for fornecida, o documento torna-se uma
%   proposta de disserta?o ou um projeto de tese. O efeito ?o mesmo para
%   qualquer uma das duas op?es.
%
% \item |single| \sep{} \defmark|onehalf| \sep{} |double|
%
%   Define o espa?mento padr? entre as linhas: simples, 1\,\textonehalf{} ou
%   duplo.
%
% \item |centertitles|
%
%   Por padr?, o texto-t?ulo das p?inas introdut?ias (resumos,
%   Agradecimentos, Sum?io, Lista de Figuras e Lista de Tabelas) ?alinhado ?%   esquerda. Com esta op?o, os t?ulos passam a ser centralizados. Tamb? ?%   afetado o t?ulo das ``Refer?cias Bibliogr?icas''. Esta op?o n? altera
%   o alinhamento do t?ulo dos cap?ulos ao longo do documento.
%
% \item \defmark|a4paper| \sep{} |letterpaper|
%
%   Define o tamanho da p?ina: A4 ou carta (``letter''). O tamanho normatizado
%   no Brasil ?A4.
%
% \item |9pt| \sep{} |10pt| \sep{} |11pt| \sep{} \defmark|12pt| \sep{} |14pt|
%   \sep{} |17pt|
%
%   Define o tamanho da fonte. 12 pontos ?o padr?. As demais alternativas
%   devem ser ignoradas na vers? final.
%
% \item |oneside| \sep{} \defmark|twoside|
%
%   Define se a impress? ser?feita apenas em um lado da folha (|oneside|) ou
%   em ambos (|twoside|). A impress? em dupla face deve ser adotada sempre que
%   poss?el, j?que reduz o consumo de papel.
%
% \item \defmark|showcover| \sep{} |hidecover|
%
%   Habilita ou desabilita a gera?o da falsa folha de rosto.
%
% \item \defmark|showtitle| \sep{} |hidetitle|
%
%   Habilita ou desabilita a gera?o da folha de rosto.
%
% \item \defmark|showfc| \sep{} |hidefc|
%
%   Habilita ou desabilita a gera?o da Ficha Catalogr?ica. (Se uma das op?es
%   |proposal| ou |project| for usada, o padr? ?|hidefc|.)
%
% \item \defmark|showapproval| \sep{} |hideapproval|
%
%   Habilita ou desabilita a gera?o da Folha de Aprova?o.
%
% \item \defmark|showabstract| \sep{} |hideabstract|
%
%   Habilita ou desabilita a gera?o das p?inas de resumo (\emph{abstract}).
%
% \item \defmark|showdedication| \sep{} |hidededication|
%
%   Habilita ou desabilita a gera?o da p?ina de dedicat?ia.
%
% \item \defmark|showack| \sep{} |hideack|
%
%   Habilita ou desabilita a gera?o da p?ina de agradecimentos.
%
% \item \defmark|showepigraph| \sep{} |hideepigraph|
%
%   Habilita ou desabilita a gera?o da p?ina com a ep?rafe.
%
% \item \defmark|showlof| \sep{} |hidelof|
%
%   Habilita ou desabilita a gera?o da Lista de Figuras.
%
% \item \defmark|showlot| \sep{} |hidelot|
%
%   Habilita ou desabilita a gera?o da Lista de Tabelas.
%
% \item \defmark|showtoc| \sep{} |hidetoc|
%
%   Habilita ou desabilita a gera?o do Sum?io (\emph{Table of Contents}).
%
% \item \defmark|showall| \sep{} |hideall|
%
%   Habilita ou desabilita a gera?o de todas as p?inas opcionais. Voc?pode
%   usar a op?o |hideall| para acelerar a compila?o do documento enquanto ele
%   est?sendo constru?o e alterar para |showall| somente na gera?o das
%   vers?es finais.
%
% \item |nomainmatter|
%
%   A classe \cls{ppgccufmg} chama automaticamente o comando |\mainmatter|
%   depois da constru?o de todas as p?inas introdut?ias. (O comando
%   |\mainmatter| reinicia a numera?o das p?inas e configura-as para o
%   formato ar?ico: 1, 2, 3, etc.) Se voc?quiser inserir manualmente outras
%   p?inas antes do corpo principal do documento de modo a manter a numera?o
%   romana (i, ii, iii, etc.), ent? use esta op?o. Neste caso, por?,
%   \emph{n? se esque?} de colocar o comando |\mainmatter| antes do primeiro
%   cap?ulo. Se voc?esquecer desse comando, todas as p?inas do seu documento
%   ser? numerados conforme a formata?o romana.
%
%   Note que esta op?o \emph{n? deve mais ser usada} para a inser?o da Lista
%   de Algoritmos ou outras listas. Para isso, consulte a op?o |beforetoc| do
%   comando |\ppgccufmg| (Se?o~\ref{sec:ppgccufmg}) e o exemplo da
%   Se?o~\ref{sec:quickintro}.
%
% \end{itemize}
%
%
% \subsection{O jeito muito f?il de usar a classe}
% \label{sec:ppgccufmg}
%
% Todas as p?inas introdut?ias --- falsa folha de rosto, folha de rosto,
% Folha de Aprova?o, resumo, resumo estendido, \emph{abstract}, dedicat?ia,
% agradecimentos, Sum?io, lista de figuras e lista de tabelas --- podem ser
% geradas por um ?nico comando: |\ppgccufmg|, que recebe todas as informa?es
% por meio de pares \meta{chave}|=|\meta{valor}. A chamada ao comando se parece
% com algo como:
%
% \begin{quote}
% \begin{raggedright}
% |\ppgccufmg{|\\
% |  title={O T?ulo da Tese},|\\
% |  authorrev={da Camara Neto, Vilar Fiuza},|\\
% |  university={Universidade Federal de Minas Gerais},|\\
% |  |\emph{demais informa?es\ldots}\\
% |}|
% \end{raggedright}
% \end{quote}
%
% \emph{Para os impacientes:} cheque os exemplos no fim desta se?o. Cheque
% tamb? o arquivo de exemplo, |exemplo.tex|, distribu?o com este pacote.
%
% Bom, agora voc?precisa da lista de todas as chaves que podem ser fornecidas
% como par?etros para o comando |\ppgccufmg|. Eis a lista:
%
% \begin{itemize}
%
% \item |title|=\marg{t?ulo}
%
%   Define o t?ulo da disserta?o ou da tese.
%
%   Ao compilar o texto, se voc?perceber que as quebras de linha que ocorrem
%   na folha de rosto criam um efeito desbalanceado, marque manualmente os
%   pontos de quebra de linha com duas contra-barras (|\\|). Para maiores
%   informa?es, veja a descri?o do comando |\title| na
%   p?ina~\pageref{macro:title}.
%
% \item |authorrev|=\marg{sobrenome,nome}
%
%   Define o nome do autor em ordem reversa (Sobrenome, Nome).
%
% \item |author|=\marg{nome}
%
%   Define o nome do autor. Se esta chave n? for usada, o nome ?montado
%   automaticamente com base na chave |authorrev|: portanto, o ideal ?usar
%   apenas |authorrev|.
%
% \item |cutter|=\marg{cutter}
%
%   Define o c?igo ``cutter'' do documento.
%
% \item |cdu|=\marg{CDU}
%
%   Define o identificador CDU do documento, fornecido pela Secretaria do Curso.
%
% \item |keywords|=\marg{chave,chave,\ldots}
%
%   Define as palavras-chave que dever? constar na Ficha Catalogr?ica,
%   separadas por v?gulas.
%
% \item |postkeywords|=\marg{chave,chave,\ldots}
%
%   Define o segundo conjunto de palavras-chave que dever? constar na Ficha
%   Catalogr?ica, separadas por v?gulas. O padr? ?apenas ``T?ulo''.
%
% \item |indexkeys|=\marg{texto}
%
%   Permite determinar exatamente o texto da indexa?o da Ficha Catalogr?ica.
%   Em geral, basta fornecer |keywords| e |postkeywords|. Esta informa?o ?%   fornecida pela Secretaria do Curso ou pela biblioteca.
%
% \item |university|=\marg{nome}
%
%   Define o nome da universidade.
%
% \item |course|=\marg{nome}
%
%   Define o nome do curso.
%
% \item |portuguesetitle|=\marg{t?ulo}\\
%   |portugueseuniversity|=\marg{nome}\\
%   |portuguesecourse|=\marg{nome}
%
%   Se voc?est?escrevendo o documento em l?gua estrangeira, voc?precisar?%   fornecer tamb? o t?ulo do documento e os nomes da universidade e do curso
%   em portugu?. Essas chaves servem para isso.
%
% \item |address|=\marg{endere?}
%
%   N? ?o endere? completo: apenas a cidade e o estado, para constarem na
%   folha de rosto.
%
% \item |date|=\marg{data}
%
%   Define a data do documento, \emph{sempre} no formato |aaaa-mm-dd| ou
%   |aaaa-mm| (ou seja, o dia n? ?obrigat?io). Note a ordem (ano-m?-dia), o
%   h?en como separador e o uso de 4 d?itos para o ano.
%
% \item |approval|=\oarg{ajuste}\oarg{escala}\marg{arquivo gr?ico}
%
%   Especifica o arquivo gr?ico que cont? a Folha de Aprova?o digitalizada.
%   Este gr?ico ?inclu?o no documento. O argumento \meta{ajuste} ?uma
%   dist?cia para deslocar a imagem para baixo (por exemplo, |-2.5cm| desloca
%   a imagem 25mm para cima), e \meta{escala} ?um fator de escala para a
%   imagem (por exemplo, |0.9| escala a imagem em 90\%).
%
%   Esta op?o pode ser fornecida mais de uma vez, caso a Folha de Aprova?o
%   ocupe mais de uma p?ina.
%
% \item |advisor|=\marg{nome}
%
%   Define o nome do orientador.
%
% \item |abstract|=\marg{t?ulo}\marg{arquivo}
%
%   Define um arquivo \texttt{.tex} que cont? o texto do resumo (ou
%   \emph{abstract}). O argumento \meta{t?ulo} refere-se ao t?ulo a constar
%   na p?ina: geralmente ``Resumo'', ``Abstract'' ou ``Resumo Estendido''.
%   \emph{N? use o comando} |\chapter| \emph{nesse arquivo.}
%
%   Voc?pode usar esta op?o diversas vezes, para especificar vers?es
%   diferentes do resumo. O mais comum ?que sejam especificadas pelo menos uma
%   vers? em portugu? e uma em ingl?. Observe os exemplos descritos a seguir.
%
% \item |dedication|=\marg{arquivo}
%
%   Define um arquivo \texttt{.tex} que cont? a dedicat?ia. \emph{N? use o
%   comando} |\chapter| \emph{nesse arquivo}, inclusive porque p?inas de
%   dedicat?ia n? apresentam t?ulo.
%
%   Voc?pode usar esta op?o diversas vezes, geralmente se quiser apresentar
%   dedicat?ias em mais de uma linguagem.
%
% \item |ack|=\oarg{t?ulo}\marg{arquivo}
%
%   Define um arquivo \texttt{.tex} que cont? os agradecimentos. Se o
%   argumento \meta{t?ulo} for omitido, ser?adotado ``Agradecimentos'' na
%   linguagem do documento. \emph{N? use o comando} |\chapter| \emph{nesse
%   arquivo.}
%
%   Voc?pode usar esta op?o diversas vezes, geralmente se quiser apresentar
%   agradecimentos em mais de uma linguagem.
%
% \item |epigraphtext|=\marg{texto}\marg{autor}
%
%   Define a ep?rafe, que ser?colocada em p?ina pr?ria.
%
% \item |beforetoc|=\marg{comandos}
%
%   Define um conjunto de comandos que deve ser executado imediatamente antes
%   da composi?o do Sum?io. Usado principalmente para a gera?o da Lista de
%   Algoritmos ou outras listas (ex.: |beforetoc={\listofalgorithms}|).
%
% \end{itemize}
%
% Eis um exemplo completo, para um documento escrito em l?gua portuguesa:
%
% \begin{quote}
% \begin{verbatim}
% \ppgccufmg{
%   title={Protocolo para Verifica?o de Erros\\em Redes Totalmente
%     Confi?eis},
%   authorrev={da Camara Neto, Vilar Fiuza},
%   university={Universidade Federal de Minas Gerais},
%   course={Ci?cia da Computa?o},
%   address={Belo Horizonte},
%   date={2005-01},
%   advisor={Adamastor Pompeu Set?bal},
%   approval={img/approvalsheet.eps},
%   abstract={Resumo}{resumo},
%   abstract=[english]{Abstract}{abstract},
%   abstract={Resumo Estendido}{resumoest},
%   dedication={dedicatoria},
%   ack={agradecimentos},
%   epigraphtext={A verdade ?o contr?io da mentira, \\
%     e a mentira ?o oposto da verdade.}{Autor desconhecido},
% }
%\end{verbatim}
% \end{quote}
%
% Note que neste exemplo a chave |abstract| foi usada v?ias vezes para definir
% vers?es diferentes do resumo.
%
% Eis um exemplo para ilustrar a estrutura do comando |\ppgccufmg| para um
% documento escrito em l?gua estrangeira (note que ?necess?io definir tamb?
% as chaves |portuguesetitle|, |portugueseuniversity| e |portuguesecourse|):
%
% \begin{quote}
% \begin{verbatim}
% \ppgccufmg{
%   title={Protocol for Error-Verification inside\\Totally Error-Free
%     Networks},
%   authorrev={da Camara Neto, Vilar Fiuza},
%   university={Federal University of Minas Gerais},
%   course={Computer Science},
%   portuguesetitle={Protocolo para Verifica?o de Erros\\em Redes
%     Totalmente Confi?eis},
%   portugueseuniversity={Universidade Federal de Minas Gerais},
%   portuguesecourse={Ci?cia da Computa?o},
%   address={Belo Horizonte},
%   date={2005-01},
%   advisor={Adamastor Pompeu Set?bal},
%   approval={img/approvalsheet.eps},
%   abstract=[brazil]{Resumo}{resumo},
%   abstract={Abstract}{abstract},
%   dedication={dedicatoria},
%   ack={agradecimentos},
%   epigraphtext={Truth and lie are opposite things.}{Unknown},
% }
%\end{verbatim}
% \end{quote}
%
%
% \subsection{O jeito f?il de usar a classe}
%
% H?outra maneira de compor as p?inas introdut?ias: pode-se esquecer o
% comando m?ico |\ppgccufmg| e chamar v?ios comandos --- alguns para definir
% as informa?es do documento, outros para construir cada p?ina introdut?ia.
% ?um procedimento menos automatizado, mas voc?ter?um controle mais preciso
% da ordem das p?inas que ser? montadas.
%
% Os comandos para definir as informa?es do documento possuem os mesmos nomes
% e argumentos das chaves descritas na se?o anterior: voc?chama os comandos
% |\title|\marg{t?ulo}, |\author|\marg{autorrev}, e assim por diante.
%
% Ap? definir todas as informa?es, voc?deve chamar comandos para construir
% cada p?ina. Esses comandos s? descritos a seguir:
%
% \begin{itemize}
%
% \item |\makecoverpage|
%
%   Comp?e a falsa folha de rosto, que cont? somente o t?ulo da obra.
%
% \item |\maketitlepage|
%
%   Comp?e a folha de rosto, onde constam os nomes do autor e do orientador, o
%   t?ulo da obra, a descri?o formal do documento, a localidade (cidade e
%   estado) e a data.
%
% \item |\makefc|
%
%   Comp?e a Ficha Catalogr?ica. Essa p?ina apresenta algumas informa?es
%   sobre o documento necess?ias para a classifica?o em bases de dados
%   bibliotec?ios. ?necess?ia somente para a vers? final de teses e
%   disserta?es (propostas e projetos n? requerem essa folha).
%
% \item |\makeapproval|
%
%   Inclui a Folha de Aprova?o. Essa folha ?fornecida pela Secretaria do
%   Curso ap? a aprova?o do documento final. Ela deve ser digitalizada e
%   armazenada em formato gr?ico (preferencialmente em formato PNG ou EPS), e
%   a op?o |approval| (ou o comando |\approval|) deve especificar o nome do
%   arquivo gr?ico. Se o nome do arquivo n? for especificado, o comando
%   |\makeapproval| gera uma p?ina com as instru?es para digitalizar a Folha
%   de Aprova?o.
%
% \item |\includeabstract|\marg{t?ulo}\marg{arquivo}
%
%   Inclui um arquivo |.tex| com o texto do resumo. \meta{t?ulo} ?o t?ulo do
%   resumo. Pode ser usado para incluir um resumo alternativo escrito em l?gua
%   estrangeira. \emph{N? use o comando} |\chapter| \emph{nesse arquivo.}
%
% \item |\includededication|\marg{arquivo}
%
%   Inclui um arquivo |.tex| com o texto da dedicat?ia. \emph{N? use o
%   comando} |\chapter| \emph{nesse arquivo.}
%
% \item |\includeack|\oarg{t?ulo}\marg{arquivo}
%
%   Inclui um arquivo |.tex| com o texto dos agradecimentos. Se \meta{t?ulo}
%   for omitido, ser?adotado ``Agradecimentos'' na linguagem do documento.
%   \emph{N? use o comando} |\chapter| \emph{nesse arquivo.}
%
% \item |\makeepigraph|
%
%   Comp?e a p?ina com a Ep?rafe. As informa?es necess?ias (texto e autor)
%   devem ser fornecidas pela op?o |epigraphtext| (ou pelo comando
%   |\epigraphtext|).
%
% \item |\listoffigures|
%
%   Comp?e a Lista de Figuras.
%
% \item |\listoftables|
%
%   Comp?e a Lista de Tabelas.
%
% \item |\tableofcontents|
%
%   Comp?e o Sum?io.
%
% \end{itemize}
%
% Ap? usar esses comandos, voc?deve usar o comando |\mainmatter|, que
% reinicia a numera?o das p?inas e configura-as para a formata?o ar?ica (1,
% 2, 3, etc.).
%
%
% \subsection{O jeito dif?il de usar a classe}
%
% Ainda n? foi implementado\ldots{} :-)
%
%
% \subsection{As refer?cias bibliogr?icas}
%
% Voc?pode usar qualquer sistema e estilo para as refer?cias bibliogr?icas:
% a classe \cls{ppgccufmg} n? obriga a ado?o de nada em particular.
%
% Entretanto, se voc?estiver usando o sistema Bib\TeX{} (que gerencia arquivos
% |.bib|), voc?pode estar se perguntando sobre o estilo a ser adotado (ou
% seja, qual deve ser o argumento para o comando |\bibliographystyle{}|?). Como
% no momento n? h?nenhum estilo oficial adotado pelo PPGCC/UFMG, aqui n?
% fornecemos um estilo fortemente baseado no tradicional |apalike|: o estilo
% \bst{ppgccufmg} (arquivo |ppgccufmg.bst|). Se quiser adot?lo, voc?deve
% colocar o arquivo |ppgccufmg.bst| em um diret?io vis?el ao sistema
% Bib\TeX{} (pode ser no mesmo diret?io que cont? o seu arquivo |.bib|).
%
% Se voc?adotar o estilo \bst{ppgccufmg}, ent? a classe \cls{ppgccufmg} prov?% um modo de gerar automaticamente a se?o de refer?cias bibliogr?icas.
% Trata-se do comando |\ppgccbibliography|. ?somente isso: n? h?necessidade
% de usar os comandos |\bibliographystyle| ou |\bibliography|. A sintaxe ?
%
% \begin{quote}
% |\ppgccbibliography|\oarg{op?es}\marg{arquivo-de-bibliografia}
% \end{quote}
% onde \meta{arquivo-de-bibliografia} ?o nome do seu arquivo |.bib| sem a
% extens?. Assim, se o seu banco de dados bibliogr?ico est?armazenado no
% arquivo |referencias.bib|, voc?deve chamar |\ppgccbibliography{referencias}|.
%
% H?algumas \meta{op?es} dispon?eis, tratando em geral da formata?o da
% bibliografia. S? elas:
%
% \begin{itemize}
%
% \item |noauthorrepeat|
%
%   Se voc?tiver uma s?ie de itens bibliogr?icos de mesma autoria, essa
%   op?o faz com que os autores apare?m somente no primeiro item: nos demais
%   itens, um travess? triplo (``---------'') aparece no lugar dos autores.
%   (Isso pode ser alterado: veja a pr?ima op?o.)
%
% \item |noauthorrepstring|
%
%   Ao usar a op?o |noauthorrepeat|, voc?pode configurar o que aparece no
%   lugar dos autores quando h?repeti?o de autoria. O padr? ?um travess?
%   triplo (ou seja, uma seq??cia de 9 h?ens, pois o \LaTeX{} converte tr?
%   h?ens em um travess?). Exemplo: |noauthorrepstring={Idem}| apresenta o
%   texto ``Idem'' no lugar dos autores repetidos.
%
% \item |nobreakitems|
%
%   Esta op?o impede a ocorr?cia de quebra de p?ina no meio de um item
%   bibligr?ico.
%
% \item |bibauthorand|
%
%   Define o texto que aparece antes do ?ltimo autor em um item bibliogr?ico.
%   O padr? ?``e'' (em portugu?) ou ``and'' (em ingl?). Exemplo: para
%   especificar um ``e'' comercial, use |bibauthorand=\&|.
%
% \item |citeauthorand|
%
%   Define o texto que aparece antes do ?ltimo autor em uma cita?o
%   bibliogr?ica. O padr? ?``e'' (em portugu?) ou ``and'' (em ingl?).
%   Exemplo: para especificar um ``e'' comercial, use |citeauthorand=\&|.
%
% \item |authorformat|
%
%   Define a formata?o da lista de autores. O padr? ?a formata?o do texto
%   padr?. Exemplo: para formatar a lista de autores em
%   versalete\footnote{Versalete: ``O que ?isso?'' ``?o famoso \emph{small
%   caps}'', ou seja, as letras min?sculas s? exibidas como mai?sculas de
%   tamanho reduzido.}, use |authorformat=\scshape|.
%
% \item |titleformat|
%
%   Define a formata?o do t?ulo. O padr? ?a formata?o do texto padr?.
%   Exemplo: para colocar o t?ulo em it?ico, use |titleformat=\itshape|.
%
% \item |btitleformat|
%
%   Define a formata?o do \emph{booktitle} para itens do tipo |@incollection|
%   e |@inproceedings|. O padr? ?|\itshape| (it?ico).
%
% \item |booktitleformat|
%
%   Define a formata?o do \emph{booktitle} para itens do tipo |@book|,
%   |@inbook|, |@manual|, |@phdthesis| e |@proceedings|. O padr? ?|\itshape|
%   (it?ico).
%
% \item |journalformat|
%
%   Define a formata?o do nome do peri?ico para itens do tipo |@article|. O
%   padr? ?|\itshape| (it?ico).
%
% \end{itemize}
%
% Certas coisas s? automaticamente traduzidas (atualmente para portugu? e
% ingl? somente). Por exemplo, se voc?tem um artigo dos autores ``Michael J.
% Brooks and Berthold K. P. Horn'', ele ser?citado como sendo de ``Brooks e
% Horn'' se o seu documento estiver em portugu? ou de ``Brooks and Horn'' se
% estiver em ingl?. Para ativar a tradu?o, voc?n? precisa especificar
% nenhuma op?o: basta informar a linguagem correta para o pacote \pkg{babel},
% ou seja:
% \begin{quote}
% |\usepackage[brazil]{babel}     % para documentos em portugu?|
% \end{quote}
% ou
% \begin{quote}
% |\usepackage[english]{babel}    % para documentos em ingl?|
% \end{quote}
%
% Finalmente, recomanda-se \emph{fortemente} a ado?o do pacote \pkg{natbib},
% para que as refer?cias tenham uma apar?cia mais profissional. Cheque a
% documenta?o desse pacote para saber do que se trata e como us?lo, em
% particular sobre os comandos |\cite| e |\citep|.
%
%
% \subsubsection{Ap?dices e anexos}
%
% H?dois ambientes feitos para a cria?o de ambientes e anexos: |appendices| e
% |attachments|, respectivamente. Ao usar ap?dices no seu documento, encapsule
% todos os cap?ulos que correspondem aos ap?dices com os comandos
% |\begin{appendices}| e |\end{appendices}|:
%
% \begin{quote}
% \begin{verbatim}
% \begin{appendices}
%
% \chapter{Um ap?dice}
% ...conte?do do ap?dice...
%
% \chapter{Outro ap?dice}
% ...conte?do do ap?dice...
%
% \end{appendices}
%\end{verbatim}
% \end{quote}
%
% Com isso, os ap?dices s? numerados com letras ar?icas em mai?sculas e o
% Sum?io insere a palavra ``Ap?dice'' antes de cada entrada.
%
% Os anexos seguem a mesma filosofia, por? os comandos s?
% |\begin{attachments}| e |\end{attachments}|. O Sum?io lista os anexos
% precedidos da palavra ``Anexo''.
%
%
% \subsection{Dicas sobre a classe \cls{memoir}}
% \label{sec:memoirtips}
%
% Como foi dito anteriormente, a classe \cls{ppgccufmg} baseia-se na classe
% \cls{memoir}, que deve estar instalada no seu sistema. Caso contr?io, o
% compilador \LaTeX{} dever?reclamar da aus?cia do arquivo |memoir.cls| e
% voc?dever?instalar essa classe. Se voc?estiver usando a distribui?o
% MiK\TeX{} para Windows, ?bastante simples: basta executar o aplicativo
% \emph{MiKTeX Package Manager} ou o \emph{Browse Packages}, selecionar
% ``memoir'' da lista de pacotes e comandar a instala?o do mesmo. Talvez voc?% precise de privil?ios de administra?o na m?uina para fazer isso: consulte
% o administrador do computador ou da rede.
%
% Sob Linux, a instala?o talvez n? seja automatizada, mas n? ?dif?il.
% Antes de mais nada, verifique se a classe j?est?dispon?el no pr?rio
% reposit?io da distribui?o (por exemplo, atrav? do |apt|, |aptitude| ou
% |synaptic| do Debian/Ubuntu): desse jeito a instala?o e as atualiza?es s?
% automatizadas.
%
% No caso de necessidade de instala?o manual: Antes de mais nada, baixe todos
% os arquivos do reposit?io CTAN: atualmente, o \emph{link} ?% \begin{quote}
% |http://www.ctan.org/tex-archive/macros/latex/contrib/memoir/|
% \end{quote}
%
% Quando os arquivos estiverem no seu computador, abra um terminal, entre no
% diret?io que cont? os arquivos baixados e execute o comando
% \begin{quote}
% |latex memoir.ins| \\
% |latex mempatch.ins|
% \end{quote}
% que criar?o arquivo |memoir.cls|. Opcionalmente, para gerar a documenta?o
% da classe \cls{memoir}, execute:
% \begin{quote}
% |latex memoir.dtx|
% \end{quote}
% Agora crie um diret?io local |texmf| no seu diret?io \emph{home}:
% \begin{quote}
% |mkdir ~/texmf/tex/latex/memoir|
% \end{quote}
% e copie para l?todos os arquivos |*.cls|, |*.sty| e |*.clo|. Finalmente,
% execute o comando |texhash|. Pronto, est?terminado!
%
% De qualquer forma, d?uma lida no arquivo |README| que vem com a classe
% \cls{memoir}.
%
% Esta se?o foi escrita com base nas informa?es contidas no arquivo |README|
% e nas instru?es presentes em |http://www.tex.ac.uk/|
% |cgi-bin/texfaq2html?label=instpackages|.
%
%
% \subsection{Algumas dicas para acelerar o seu trabalho}
%
% \subsubsection{Pacotes essenciais}
%
% Provavelmente voc?vai querer os seguintes comandos no pre?bulo do seu
% documento:
%
% \begin{quote}
% \begin{verbatim}
% \usepackage[brazil]{babel}
% \usepackage[latin1]{inputenc}
% \usepackage[T1]{fontenc}
% \usepackage[square]{natbib}
%\end{verbatim}
% \end{quote}
%
% Se tamb? quiser gr?icos (quem n? quer?):
%
% \begin{quote}
% \begin{verbatim}
% \usepackage{graphicx}
%\end{verbatim}
% \end{quote}
%
% \subsubsection{\emph{Links} clic?eis}
%
% Se voc?quiser que o arquivo |.pdf| gerado possua \emph{links} clic?eis em
% todas as refer?cias a se?es, figuras, tabelas, p?inas, cita?es, etc.:
%
% \begin{quote}
% \begin{verbatim}
% \usepackage[a4paper,
%   portuguese,
%   bookmarks=true,
%   bookmarksnumbered=true,
%   linktocpage,
%   colorlinks=true,
%   citecolor=black,
%   urlcolor=black,
%   linkcolor=black,
%   filecolor=black,
%   ]{hyperref}
%\end{verbatim}
% \end{quote}
%
% As op?es \ldots|color| padronizam a cor preta para os \emph{links}
% clic?eis, o que ?a melhor alternativa para a impress?.
%
% \subsection{Poss?eis problemas e suas solu?es}
%
% \subsubsection{Pacote \pkg{algorithm}, ou ``Command \texttt{\textbackslash
% newfloat} already defined''}
%
% Um problema freq?ente ?causado por uma incompatibilidade entre a classe
% \cls{memoir} e o pacote \pkg{algorithm}. Esse problema ?facilmente
% identificado pela ocorr?cia do seguinte erro:
% \begin{quote}
% \begin{verbatim}
% ! LaTeX Error: Command \newfloat already defined.
%                Or name \end... illegal, see p.192 of the manual.
%
% See the LaTeX manual or LaTeX Companion for explanation.
% Type  H <return>  for immediate help.
%\end{verbatim}
% \end{quote}
%
% A solu?o ?simples: basta colocar o comando |\let\newfloat=\undefined| antes
% de |\includepackage{algorithm}|. Dessa forma, o seu pre?bulo deve ficar mais
% ou menos como segue:
% \begin{quote}
% \begin{verbatim}
% \usepackage...
% \usepackage...
% \let\newfloat=\undefined
% \usepackage{algorithm}
% \usepackage...
% \usepackage...
%\end{verbatim}
% \end{quote}
%
% Espero que esse ``bug'' na classe \cls{memoir} seja resolvido em breve. Por
% enquanto, esse ``hack'' funciona bem e nunca ouvi falar de algum efeito
% adverso. (Se voc?descobrir algum, por favor me avise!)
%
% \subsubsection{``\texttt{\textbackslash cftchapterformatpnum} undefined''}
%
% Se o seguinte erro ocorrer durante a compila?o:
% \begin{quote}
% \begin{verbatim}
% ppgccufmg.cls:xxxx:\cftchapterformatpnum undefined.
% \renewcommand*{\cftchapterformatpnum}
%\end{verbatim}
% \end{quote}
% \ldots ent? o seu sistema \LaTeX{} precisa ser atualizado, principalmente os
% arquivos da classe \cls{memoir}. Verifique a Subse?o~\ref{sec:memoirtips}
% para maiores informa?es.
%
%
% \section{C?igo-fonte}
%
% Esta se?o documenta o c?igo-fonte do arquivo |ppgccufmg.cls|. Voc?n?
% precisa ler esta se?o: ela serve como refer?cia para os desenvolvedores do
% pacote.
%
%
% \subsection*{Pr?requisitos e comandos gerais}
%
% A classe \cls{ppgccufmg} requer alguns pacotes, portanto eles ser?
% inclu?os:
%
%    \begin{macrocode}
\RequirePackage{keyval}
\RequirePackage{setspace}
\RequirePackage{textpos}
\RequirePackage{etextools}
%    \end{macrocode}


% \DescribeMacro{\ppgcc@classname}%
% Defini?o do nome da classe, usada em alguns pontos do c?igo:
%
%    \begin{macrocode}
\def\ppgcc@classname{ppgccufmg}
%    \end{macrocode}

% As op?es de classe s? gerenciadas por diretivas do tipo |\newif|, que s?
% declaradas e inicializadas a seguir:
%
%    \begin{macrocode}
%% Op?es gerais:

\newif\ifppgcc@phd                 % Tese de doutorado?
\newif\ifppgcc@msc                 % Disserta?o de mestrado?
\newif\ifppgcc@proposal            % Op?o de proposta/projeto?
\newif\ifppgcc@showcover           % Compor a falsa folha de rosto?
\newif\ifppgcc@showtitle           % Compor a folha de rosto?
\newif\ifppgcc@showfc              % Compor a Ficha Catalogr?ica?
\newif\ifppgcc@showapproval        % Incluir a Folha de Aprova?o?
\newif\ifppgcc@showabstract        % Compor as p?inas de resumo/abstract?
\newif\ifppgcc@showdedication      % Compor a dedicat?ia?
\newif\ifppgcc@showack             % Compor os agradecimentos?
\newif\ifppgcc@showepigraph        % Compor a Ep?rafe?
\newif\ifppgcc@showlof             % Compor a Lista de Figuras?
\newif\ifppgcc@showlot             % Compor a Lista de Tabelas?
\newif\ifppgcc@showtoc             % Compor o Sum?io?
\newif\ifppgcc@extraporttitlepage  % Exibir tamb? uma folha de t?ulo
                                   % em portugu??
\newif\ifppgcc@putmainmatter       % Chamar automaticamente \mainmatter depois
                                   % das p?inas introdut?ias?
\newif\ifppgcc@hasapprovalpages    % H?pelo menos uma Folha de Aprova?o?

\newif\if@selfdoc
\ifx\ppgcc@selfdoc\undefined\else\@selfdoctrue\fi
\if@selfdoc\input{ppgccufmgselfdoc}\fi

\def\ppgcc@showall{%
  \ppgcc@showcovertrue
  \ppgcc@showtitletrue
  \ifppgcc@proposal\ppgcc@showfcfalse\else\ppgcc@showfctrue\fi
  \ppgcc@showapprovaltrue
  \ppgcc@showabstracttrue
  \ppgcc@showdedicationtrue
  \ppgcc@showacktrue
  \ppgcc@showepigraphtrue
  \ppgcc@showloftrue
  \ppgcc@showlottrue
  \ppgcc@showtoctrue
}

\def\ppgcc@hideall{%
  \ppgcc@showcoverfalse
  \ppgcc@showtitlefalse
  \ppgcc@showfcfalse
  \ppgcc@showapprovalfalse
  \ppgcc@showabstractfalse
  \ppgcc@showdedicationfalse
  \ppgcc@showackfalse
  \ppgcc@showepigraphfalse
  \ppgcc@showloffalse
  \ppgcc@showlotfalse
  \ppgcc@showtocfalse
}

\ppgcc@proposalfalse
\ppgcc@showall
\ppgcc@putmainmattertrue
\ppgcc@hasapprovalpagesfalse

%% Op?es das refer?cias bibliogr?icas:

\newif\if@bibnorepauthor  % Substituir autorias repetidas por linha comprida?
\newif\if@nobreakitems    % Impedir quebra de p?ina no meio de um item?

\@bibnorepauthorfalse
\@nobreakitemsfalse
%    \end{macrocode}

% \DescribeMacro{\ppgcc@gobblespaces}%
% Esta ?uma pequena macro auxiliar para remover os espa?s iniciais de uma
% string.
%
%    \begin{macrocode}
\def\ppgcc@gobblespaces #1{#1}%
%    \end{macrocode}

% \DescribeMacro{\BreakableUppercase}%
% |\BreakableUppercase| prov?um ``hack'' para |\MakeUppercase| que permite a
% ocorr?cia de quebras de linha (|\\|) em seu argumento. Assim como o comando
% |\MakeUppercase|, este comando ?chamado com um argumento obrigat?io:
% |\BreakableUppercase| \marg{t?ulo}.
%
%    \begin{macrocode}
\def\BreakableUppercase#1{%
  \bgroup
  \let\ppgcc@prevdbs=\\%
  \def\\{\protect\ppgcc@prevdbs}%
  \MakeUppercase{#1}%
  \egroup
}
%    \end{macrocode}

% \DescribeMacro{\ppgcc@defspacing}%
% Esta macro armazena o espa?mento padr? entre linhas, que pode ser escolhida
% pelo usu?io por meio de uma das seguintes op?es de classe: |single|,
% |onehalf| ou |double|.
%
%    \begin{macrocode}
\newcommand{\ppgcc@defspacing}{\onehalfspace}
%    \end{macrocode}

% \DescribeMacro{\ppgcc@redefchaptitlefont}%
% Esta macro redefine a formata?o do texto produzido pelo comando |\chapter|.
% O ?nico objetivo ?redefinir |\chaptitlefont| (consulte o manual do
% \cls{memoir}) para incluir o comando |\centering| se a op?o de classe
% |centertitles| for especificada.
%
%    \begin{macrocode}
\def\ppgcc@redefchaptitlefont{}
%    \end{macrocode}

% \DescribeMacro{\authorrev}%
% \DescribeMacro{\advisor}%
% \DescribeMacro{\coadvisor}%
% \DescribeMacro{\university}%
% \DescribeMacro{\course}%
% \DescribeMacro{\date}%
% \DescribeMacro{\address}%
% Este conjunto de comandos define algumas informa?es sobre o documento. Cada
% comando recebe um argumento obrigat?io, que ?a informa?o associada ao
% comando.
%
%    \begin{macrocode}
\let\ppgcc@coadvisorname\empty

\newcommand*{\authorrev}[1]{\gdef\ppgcc@authorrev{#1}}     % Nome rev. do autor
\newcommand*{\advisor}[1]{\gdef\ppgcc@advisorname{#1}}     % Nome do orientador
\newcommand*{\coadvisor}[1]{\gdef\ppgcc@coadvisorname{#1}} % Nome do coorientador
\newcommand*{\university}[1]{\gdef\ppgcc@university{#1}}   % Nome da universidade
\newcommand*{\course}[1]{\gdef\@course{#1}}                % Nome do curso
\newcommand*{\address}[1]{\gdef\@address{#1}}              % Endere? (cidade/UF)
%    \end{macrocode}
%
% \DescribeMacro{\cutter}%
% \DescribeMacro{\cdu}%
% \DescribeMacro{\indexkeys}%
%
%    \begin{macrocode}
\newcommand*{\fcuniversity}[1]                             % Nome da universidade
  {\gdef\ppgcc@fcuniversity{#1}}                           % para a F.C.
\newcommand*{\cutter}[1]{\gdef\ppgcc@cutter{#1}}           % C?igo "cutter"
\newcommand*{\cdu}[1]{\gdef\ppgcc@cdu{#1}}                 % C?igo CDU
\newcommand*{\indexkeys}[1]{\gdef\ppgcc@indexkeys{#1}}     % Chaves para indexa?o

\indexkeys{\ppgccexpandlistoffckeywords{} \ppgccexpandlistoffcpostkeywords}

\fcuniversity{\ppgcc@university}

\let\ppgcc@origdatecmd=\date
\renewcommand{\date}[1]{%                                  % Data (aaaa-mm ou aaaa-mm-dd)
  \ppgcc@origdatecmd{#1}\ppgcc@parsedate#1---\relax}
%    \end{macrocode}

% \DescribeMacro{\coursedegree}%
% Este comando define a titula?o buscada pelo autor. N? ?obrigat?io chamar
% este comando, pois um padr? ?automaticamente configurado --- ``Mestre'',
% ``Doutor'' ou uma tradu?o correspondente --- de acordo com a op?o de classe
% (|msc| ou |phd|) e a linguagem corrente (selecionada por meio do pacote
% |babel|).
%
%    \begin{macrocode}
\newcommand*{\coursedegree}[1]{\gdef\@degree{#1}}          % Titula?o
%    \end{macrocode}
%
% \DescribeMacro{\title}%
% \label{macro:title}%
% Este comando define o t?ulo do documento, que nas folhas de rosto e na Ficha
% Catalogr?ica.
%
% Quando o t?ulo do documento ?longo e n? cabe em uma ?nica linha (nas
% folhas de rosto), a quebra autom?ica de linha pode causar um
% desbalanceamento do texto entre as linhas. Por exemplo, se o t?ulo for
% ``Este ?o t?ulo comprido da minha tese'', o texto pode ser quebrado da
% seguinte maneira:
% \begin{center}
% Este ?o t?ulo comprido da minha\\tese
% \end{center}
%
% Nesses casos, pode-se especificar manualmente as quebras de linha com duas
% contra-barras. Por exemplo: |\title{Este ?o t?ulo comprido\\da minha tese}|
% gera uma quebra mais balanceada:
%
% \begin{center}
% Este ?o t?ulo comprido\\da minha tese
% \end{center}
%
% O uso do comando de quebra de linha (|\\|) n? ?necess?io se o autor
% estiver satisfeito com o balanceamento gerado automaticamente.
%
%    \begin{macrocode}
\renewcommand*{\title}[1]                                  % T?ulo
  {\gdef\@title{#1}}
%    \end{macrocode}

% \DescribeMacro{\portuguesetitle}%
% \DescribeMacro{\portugueseuniversity}%
% \DescribeMacro{\portuguesecourse}%
% Para documentos n? escritos em portugu?, esses comandos especificam as
% informa?es correspondentes em l?gua portuguesa:
%    \begin{macrocode}
\newcommand*{\portuguesetitle}[1]                          % T?ulo em portugu?
  {\gdef\@portuguesetitle{#1}}
\newcommand*{\portugueseuniversity}[1]                     % Nome da univers. em pt.
  {\gdef\@portugueseuniversity{#1}}
\newcommand*{\portuguesecourse}[1]                         % Nome do curso em port.
  {\gdef\@portuguesecourse{#1}}
%    \end{macrocode}

% \DescribeMacro{\fckeywords}%
% Este comando define a lista de palavras-chave que aparece na Ficha
% Catalogr?ica, separadas por v?gulas. Exemplo:
% \begin{quote}
%   |\fckeywords{Vis? Computacional, Computa?o Gr?ica, Renderiza?o}|
% \end{quote}
%    \begin{macrocode}
\def\ppgccexpandlistoffckeywords@do[#1]#2{%
  \ifnum#1>0\relax\space\fi
  \@arabic{\numexpr1+#1\relax}.~#2.%
  }
\newcommand\ppgccexpandlistoffckeywords{%
  \@ifundefined{ppgcc@fc@keywords}{%
    \ClassError{\ppgcc@classname}{%
      As palavras-chave da Ficha Catalografica devem ser\MessageBreak
      obrigatoriamente fornecidas pela opcao `keywords' do\MessageBreak
      comando \string\ppgccufmg\space(ou pelo comando \string\fckeywords)}{\@ehd}%
    \endinput
  }{}%
  \csvloop+[\ppgccexpandlistoffckeywords@do]{\ppgcc@fc@keywords}%
  }

\newcommand{\fckeywords}[1]{\gdef\ppgcc@fc@keywords{#1}}
%    \end{macrocode}

% \DescribeMacro{\fcpostkeywords}%
% Este comando define a segunda lista de palavras-chave que aparece na Ficha
% Catalogr?ica, separadas por v?gulas. Exemplo:
% \begin{quote}
%   |\fcpostkeywords{Orientador, T?ulo}|
% \end{quote}
%    \begin{macrocode}
\def\ppgccexpandlistoffcpostkeywords@do[#1]#2{%
  \ifnum#1>0\relax\space\fi
  \@Roman{\numexpr1+#1\relax}.~#2.%
  }
\newcommand\ppgccexpandlistoffcpostkeywords{%
  \@ifundefined{ppgcc@fc@postkeywords}{%
    \ClassError{\ppgcc@classname}{%
      As palavras-chave da Ficha Catalografica devem ser\MessageBreak
      obrigatoriamente fornecidas pela opcao `postkeywords' do\MessageBreak
      comando \string\ppgccufmg\space(ou pelo comando \string\fcpostkeywords)}{\@ehd}%
    \endinput
  }{}%
  \csvloop+[\ppgccexpandlistoffcpostkeywords@do]{\ppgcc@fc@postkeywords}%
  }

\newcommand{\fcpostkeywords}[1]{\gdef\ppgcc@fc@postkeywords{#1}}
%    \end{macrocode}
%
% \DescribeMacro{\approval}%
% Este comando define o arquivo gr?ico que cont? a Folha de Aprova?o
% digitalizada. (A Folha de Aprova?o ?fornecida pela Secretaria do Curso ap?
% a aprova?o da vers? final do documento.) Pode-se especificar o diret?io
% e/ou a extens? do arquivo. Se o comando n? for usado, uma p?ina com as
% instru?es de digitaliza?o ser?inclu?a no documento. Lembre-se de que o
% formato do arquivo (dado por sua extens?) deve ser compat?el com o sistema
% \LaTeX{} em uso: por exemplo, o formato de arquivo |.eps| n? ?compat?el
% com a gera?o de arquivos |.pdf| pelo |pdflatex|.
%
%    \begin{macrocode}
\newlength{\ppgcc@approvalwidth}
\newlength{\ppgcc@approvalvertadjust}

\let\@ppgcc@approvallist=\empty
\newcommand{\ppgcc@addapproval}[1]{%
  \expandafter\gdef\expandafter\@ppgcc@approvallist\expandafter{%
    \@ppgcc@approvallist{#1}}%
  \ppgcc@hasapprovalpagestrue
  }
\newcommand{\ppgcc@doapprovallist}[1]{%
  % Para cada folha de aprova?o, define localmente as vari?eis
  % \ppgcc@approvalvertadjust, \ppgcc@approvalwidth e \ppgcc@approval e em
  % seguida chama o comando #1.
  \begingroup
  \expandafter\@tfor\expandafter\@ppgcc@tempa\expandafter:\expandafter=%
    \@ppgcc@approvallist\do{%
    \expandafter\ppgcc@approval@opt\@ppgcc@tempa\@
    #1
    }%
  \endgroup
  }

\newcommand*{\ppgcc@approval@opt}[1][0pt]{%
  \setlength{\ppgcc@approvalvertadjust}{#1}%
  \ppgcc@approval@opt@}
\newcommand{\ppgcc@approval@opt@}[1][1]{%
  \setlength{\ppgcc@approvalwidth}{#1\textwidth}%
  \ppgcc@approval@opt@@}
\def\ppgcc@approval@opt@@#1\@{%
  \def\ppgcc@approval{#1}%
  }

\newcommand*{\approval}[1][0pt]{%
  \begingroup
  \def\@ppgcc@tempa{[#1]}%
  \approval@}
\newcommand{\approval@}[2][1]{%
  \expandafter\def\expandafter\@ppgcc@tempa\expandafter{\@ppgcc@tempa[#1]{#2}}%
  \expandafter\ppgcc@addapproval\expandafter{\@ppgcc@tempa}%
  \endgroup
  }
%    \end{macrocode}
%
% \DescribeMacro{\epigraphtext}%
% Este comando define o texto da ep?rafe, que ser?inserido em uma p?ina
% pr?ria. Recebe dois argumentos: |\epigraphtext|\marg{texto}\marg{autor}.
%
%    \begin{macrocode}
\newcommand*{\epigraphtext}[2]{%
  \gdef\ppgcc@epigraphtext{#1}\gdef\ppgcc@epigraphauthor{#2}%
}
%    \end{macrocode}

% V?ios comandos s? criados para lidar com as op?es de classe:
%
%    \begin{macrocode}
\DeclareOption{msc}{\ppgcc@phdfalse\ppgcc@msctrue}
\DeclareOption{phd}{\ppgcc@phdtrue\ppgcc@mscfalse}
\DeclareOption{proposal}{%
  \ppgcc@proposaltrue\ppgcc@showfcfalse\ppgcc@showapprovalfalse}
\DeclareOption{project}{%
  \ppgcc@proposaltrue\ppgcc@showfcfalse\ppgcc@showapprovalfalse}

\DeclareOption{single}{\renewcommand{\ppgcc@defspacing}{\singlespace}}
\DeclareOption{onehalf}{\renewcommand{\ppgcc@defspacing}{\onehalfspace}}
\DeclareOption{double}{\renewcommand{\ppgcc@defspacing}{\doublespace}}

\DeclareOption{hideall}{\ppgcc@hideall}
\DeclareOption{hidecover}{\ppgcc@showcoverfalse}
\DeclareOption{hidetitle}{\ppgcc@showtitlefalse}
\DeclareOption{hidefc}{\ppgcc@showfcfalse}
\DeclareOption{hideapproval}{\ppgcc@showapprovalfalse}
\DeclareOption{hideabstract}{\ppgcc@showabstractfalse}
\DeclareOption{hidededication}{\ppgcc@showdedicationfalse}
\DeclareOption{hideack}{\ppgcc@showackfalse}
\DeclareOption{hideepigraph}{\ppgcc@showepigraphfalse}
\DeclareOption{hidelof}{\ppgcc@showloffalse}
\DeclareOption{hidelot}{\ppgcc@showlotfalse}
\DeclareOption{hidetoc}{\ppgcc@showtocfalse}

\DeclareOption{showall}{\ppgcc@showall}
\DeclareOption{showcover}{\ppgcc@showcovertrue}
\DeclareOption{showtitle}{\ppgcc@showtitletrue}
\DeclareOption{showfc}{\ppgcc@showfctrue}
\DeclareOption{showapproval}{\ppgcc@showapprovaltrue}
\DeclareOption{showabstract}{\ppgcc@showabstracttrue}
\DeclareOption{showdedication}{\ppgcc@showdedicationtrue}
\DeclareOption{showack}{\ppgcc@showacktrue}
\DeclareOption{showepigraph}{\ppgcc@showepigraphtrue}
\DeclareOption{showlof}{\ppgcc@showloftrue}
\DeclareOption{showlot}{\ppgcc@showlottrue}
\DeclareOption{showtoc}{\ppgcc@showtoctrue}

\DeclareOption{a4paper}{\def\ppgcc@papersize{a4paper}}
\DeclareOption{letterpaper}{\def\ppgcc@papersize{letterpaper}}

\DeclareOption{9pt}{\def\ppgcc@ptsize{9pt}}
\DeclareOption{10pt}{\def\ppgcc@ptsize{10pt}}
\DeclareOption{11pt}{\def\ppgcc@ptsize{11pt}}
\DeclareOption{12pt}{\def\ppgcc@ptsize{12pt}}
\DeclareOption{14pt}{\def\ppgcc@ptsize{14pt}}
\DeclareOption{17pt}{\def\ppgcc@ptsize{17pt}}

\DeclareOption{nomainmatter}{\ppgcc@putmainmatterfalse}

\DeclareOption{centertitles}{%
   \def\ppgcc@redefchaptitlefont{%
    \expandafter\renewcommand\expandafter{\expandafter\chaptitlefont
    \expandafter}\expandafter{\chaptitlefont\centering}%
  }%
}

\DeclareOption*{\PassOptionsToClass{\CurrentOption}{memoir}}
%    \end{macrocode}

% As op?es padr? s? configuradas a seguir:
%
%    \begin{macrocode}
\ExecuteOptions{a4paper,12pt,phd,onehalf,showall}
\ProcessOptions
%    \end{macrocode}

% \DescribeMacro{\ppgcc@setdoctype}%
% Esta macro checa algumas op?es do documento e ajusta os registros internos
% |\@degree| e |\@documenttype|. Esta macro ?chamada pela macro
% |\ppgcc@selectlanguage|, que ?chamada durante o comando |\begin{document}|.
%
%    \begin{macrocode}
\def\ppgcc@setdoctype{%
  \ifppgcc@phd
    % Esta ?uma tese de doutorado
    \coursedegree{\ppgcc@phd}
    \ifppgcc@proposal
      \gdef\@documenttype{\ppgcc@phddocproj}%
      \ppgcc@set@phddocproj
    \else
      \gdef\@documenttype{\ppgcc@phddoc}%
      \ppgcc@set@phddoc
    \fi
  \else
    % Esta ?uma disserta?o de mestrado
    \coursedegree{\ppgcc@msc}
    \ifppgcc@proposal
      \gdef\@documenttype{\ppgcc@mscdocprop}%
      \ppgcc@set@mscdocprop
    \else
      \gdef\@documenttype{\ppgcc@mscdoc}%
      \ppgcc@set@mscdoc
    \fi
  \fi
}
%    \end{macrocode}

% Agora a p?ina ?formatada. Aqui est? as op?es sobre o tamanho da p?ina,
% tamanho padr? da fonte, pagina?o, margens da p?ina, etc. Todo o trabalho ?% feito pela classe \cls{memoir}, o que simplifica bastante o gerenciamento do
% \emph{layout}. Consulte a Se?o~\ref{sec:memoirtips} sobre o uso da classe
% \cls{memoir}.
%
%    \begin{macrocode}
\LoadClass[\ppgcc@papersize,\ppgcc@ptsize]{memoir}

\def\ppgcc@outermargin{2.5cm}
\def\ppgcc@innermargin{3.0cm}
\def\ppgcc@topmargin{2.0cm}
\def\ppgcc@bottommargin{2.0cm}
\def\ppgcc@toptextmargin{3.5cm}
\def\ppgcc@bottomtextmargin{3.0cm}
\def\ppgcc@headerheight{1.5cm}
\def\ppgcc@footerheight{1.0cm}

\setlrmarginsandblock{\ppgcc@innermargin}{\ppgcc@outermargin}{*}
\setulmarginsandblock{\ppgcc@toptextmargin}{\ppgcc@bottomtextmargin}{*}
\setheadfoot{\baselineskip}{\ppgcc@footerheight}
\setheaderspaces{\ppgcc@topmargin}{*}{*}
\checkandfixthelayout[fixed]
%    \end{macrocode}

% Configura?o da profundidade do Sum?io e da numera?o de se?es:
%
%    \begin{macrocode}
\maxsecnumdepth{subsubsection}
\setcounter{tocdepth}{2}
\setcounter{secnumdepth}{3}
%    \end{macrocode}

% Remove a defini?o de |\@author|, para que seja detectada a falta de
% defini?o do autor:
%
%    \begin{macrocode}
\let\@author=\undefined
%    \end{macrocode}


% \subsection*{Rotinas auxiliares e de checagem}


% \DescribeMacro{\ppgcc@deflanguage}%
% Checa a sintaxe do campo |authorrev|.
%
%    \begin{macrocode}
\def\ppgcc@@checkauthorrev#1,#2,#3,#4\@{%
  \def\ppgcc@temp{#2}\ifx\ppgcc@temp\empty
    \ClassError{\ppgcc@classname}{O texto de authorrev deve
      obrigatoriamente\MessageBreak
      ser fornecido no formato `Sobrenome, Nome'}{\@ehd}%
    \expandafter\endinput
  \fi
  \@ifundefined{@author}{%
    \edef\@author{\ppgcc@gobblespaces#2\space\ppgcc@gobblespaces#1}%
  }{}%
}
\def\ppgcc@checkauthor{%
  \ifx\@author\undefined
    \ifx\ppgcc@authorrev\undefined\else
      \expandafter\ppgcc@@checkauthorrev\ppgcc@authorrev,,,\@\relax
    \fi
  \fi
  \ifx\@author\undefined
    \ClassError{\ppgcc@classname}{O autor nao foi informado (use a chave
      authorrev ou author)}{\@ehd}%
    \expandafter\endinput
  \fi
}

%%%\ppgcc@checkauthorrev\ppgcc@authorrev,,,\@
%    \end{macrocode}


% \subsection*{Rotinas de tradu?o}
%
% \DescribeMacro{\ppgcc@langgroup}%
% Agora s? definidas as rotinas de tradu?o. Tenta-se usar a linguagem
% configurada para o pacote \pkg{babel}; se esse pacote n? for usado, a
% linguagem padr? ?|brazil|.
%
%    \begin{macrocode}
\def\ppgcc@langgroup{\@ifundefined{languagename}{brazil}{\languagename}}
%    \end{macrocode}

% \DescribeMacro{\ppgcc@deflanguage}%
% \DescribeMacro{\ppgcc@selectlanguage}%
% O par de comandos |\ppgcc@deflanguage|\marg{linguagem} e
% |\ppgcc@selectlanguage| trabalham em conjunto para configurar a tabela de
% tradu?es a ser usada na composi?o do texto. O primeiro comando recebe o
% nome da linguagem como argumento; o segundo efetivamente configura o sistema.
% Atualmente, somente as linguagens |brazil| e |english| s? reconhecidas. Caso
% uma outra linguagem seja especificada, uma advert?cia ?disparada e o padr?
% |brazil| ?adotado.
%
%    \begin{macrocode}
\def\ppgcc@deflanguage#1{%
  \@ifundefined{ppgcc@lang@#1}{%
    \ClassWarning{\ppgcc@classname}{Linguagem nao definida: #1^^J
    Adotando a linguagem padrao `brazil'.^^J}%
    \def\ppgcc@langgroup{brazil}}
    {\def\ppgcc@langgroup{#1}}%
}
\def\ppgcc@selectlanguage{%
  \csname ppgcc@lang@\ppgcc@langgroup\endcsname\ppgcc@setdoctype
}
%    \end{macrocode}

% Os seguintes comandos e |\newif|s auxiliares s? definidos para ajudar nas
% rotinas de tradu?o.
%
%    \begin{macrocode}
\newif\ifppgcc@dateerror\ppgcc@dateerrortrue

\newcommand{\ppgcc@formatdate}[1][]{%
  \begingroup
    \def\ppgcc@templang{#1}%
    \ifx\ppgcc@templang\empty\let\ppgcc@templang\ppgcc@langgroup\fi
    \@ifundefined{ppgcc@year}{??}{\csname ppgcc@expanddate@\ppgcc@templang\endcsname}%
  \endgroup
}
\def\ppgcc@parsedate#1-#2-#3-#4\relax{%
  \ppgcc@dateerrortrue
  \gdef\ppgcc@year{\number#1}\gdef\ppgcc@month{\number#2}%
  \gdef\ppgcc@day{#3}\ifx\ppgcc@day\empty\else\gdef\ppgcc@day{\number#3}\fi
  \ifx\ppgcc@month\empty\else\ifnum#1>99\relax
    \ppgcc@dateerrorfalse
  \fi\fi
  \ifppgcc@dateerror
    \ClassError{\ppgcc@classname}{Formato invalido de data.^^J%
    Ao usar a opcao `date=' para o comando \string\ppgccufmg\space ou o
    comando \string\date\string{...\string},^^J%
    sempre forneca a data no formato aaaa-mm-dd ou aaaa-mm^^J%
    (note a ordem ano-mes-dia e o hifen como separador)}{\@ehd}%
    \expandafter\endinput
  \fi
}
%    \end{macrocode}

% \DescribeMacro{\ppgcc@lang@brazil}%
% \DescribeMacro{\ppgcc@lang@english}%
% Agora s? definidas as tabelas de tradu?o. Voc?pode melhorar a classe e
% adicionar novas tabelas, desde que \emph{todos} os dados sejam traduzidos!
%
%    \begin{macrocode}
\def\ppgcc@expandmonth@brazil#1{%
  \ifcase\ppgcc@month\or
  janeiro\or fevereiro\or mar\c{c}o\or abril\or maio\or junho\or
  julho\or agosto\or setembro\or outubro\or novembro\or dezembro\fi
}
\def\ppgcc@expanddate@brazil{%
  \edef\text@month{\ppgcc@expandmonth@brazil{\ppgcc@month}}%
  \ifx\ppgcc@day\empty\expandafter\MakeUppercase\text@month
    \else \two@digits\ppgcc@day\ de\ \text@month\fi
  \ de\ \ppgcc@year
}
\def\ppgcc@lang@brazil{%
  \ppgcc@extraporttitlepagefalse

  \def\@deflang@title{%
    \@ifundefined{@portuguesetitle}{\@title}{\@portuguesetitle}}
  \def\@deflang@university{%
    \@ifundefined{@portugueseuniversity}{\ppgcc@university}%
    {\@portugueseuniversity}}
  \def\@deflang@course{%
    \@ifundefined{@portuguesecourse}{\@course}{\@portuguesecourse}}

  \def\ppgcc@msc{Mestre}
  \def\ppgcc@phd{Doutor}
  \def\ppgcc@mscdoc{Disserta\c{c}\~{a}o}
  \def\ppgcc@mscdocprop{Proposta de disserta\c{c}\~{a}o}
  \def\ppgcc@phddoc{Tese}
  \def\ppgcc@phddocproj{Projeto de tese}
  \def\ppgcc@advisor{Orientador}
  \def\ppgcc@coadvisor{Coorientador}
  \def\ppgcc@docdescription{%
    \@documenttype{} apresentad\ppgcc@doctype@gender{} ao Programa de
    P\'{o}s\discretionary{-}{-}{-}Gradua\c{c}\~{a}o em \@deflang@course{} do
    Instituto de Ci\^{e}ncias Exatas da \@deflang@university{} como requisito
    parcial para a obten\c{c}\~{a}o do grau de \@degree{} em \@deflang@course.}
  \def\ppgcc@acknowledgments{Agradecimentos}
  \def\ppgcc@keywordsname{Palavras-chave}

  \def\ppgcc@set@phddocproj{\def\ppgcc@doctype@gender{o}}
  \def\ppgcc@set@phddoc{\def\ppgcc@doctype@gender{a}}
  \def\ppgcc@set@mscdocprop{\def\ppgcc@doctype@gender{a}}
  \def\ppgcc@set@mscdoc{\def\ppgcc@doctype@gender{a}}

  \def\ppgccbibauthorsep{;}
  \def\ppgccbibauthorlastsep{}
  \def\ppgccbibauthorand{\&}
  \def\ppgccciteauthorand{\&}
  \def\ppgccbibinstring{Em}
  \def\ppgccbibpagestring{p.\@}
  \def\ppgccbibpagesstring{pp.\@}
  \def\ppgccbibeditorstring{editor}
  \def\ppgccbibeditorsstring{editores}
  \def\ppgccbibeditionstring{edi\c{c}\~{a}o}
  \def\ppgccbibchapterstring{cap\'{i}tulo}
  \def\ppgccbibtechreportstring{Relat\'{o}rio t\'{e}cnico}
  \def\ppgccbibmscstring{Disserta\c{c}\~{a}o de mestrado}
  \def\ppgccbibphdstring{Tese de doutorado}

  \def\ppgcc@appendixname{Ap\^{e}ndice}
  \def\ppgcc@attachmentname{Anexo}
}

\def\ppgcc@expandmonth@english#1{%
  \ifcase\ppgcc@month\or
  january\or february\or march\or april\or may\or june\or
  july\or august\or september\or october\or november\or december\fi
}
\def\ppgcc@expanddate@english{%
  \edef\text@month{\ppgcc@expandmonth@english{\ppgcc@month}}%
  \expandafter\MakeUppercase\text@month
  \ifx\ppgcc@day\empty\else\ \ppgcc@day,\fi
  \ \ppgcc@year
}
\def\ppgcc@lang@english{%
  \ppgcc@extraporttitlepagetrue

  \def\@deflang@title{\@title}
  \def\@deflang@university{\ppgcc@university}
  \def\@deflang@course{\@course}

  \def\ppgcc@msc{Master}
  \def\ppgcc@phd{Doctor}
  \def\ppgcc@mscdoc{Dissertation}
  \def\ppgcc@mscdocprop{Dissertation proposal}
  \def\ppgcc@phddoc{Thesis}
  \def\ppgcc@phddocproj{Thesis project}
  \def\ppgcc@advisor{Advisor}
  \def\ppgcc@coadvisor{Co-Advisor}
  \def\ppgcc@docdescription{%
    \@documenttype{} presented to the Graduate Program in \@course{} of the
    \ppgcc@university{} in partial fulfillment of the requirements for the
    degree of \@degree{} in \@course.}
  \def\ppgcc@acknowledgments{Acknowledgments}
  \def\ppgcc@keywordsname{Keywords}

  \let\ppgcc@set@phddocproj\relax
  \let\ppgcc@set@phddoc\relax
  \let\ppgcc@set@mscdocprop\relax
  \let\ppgcc@set@mscdoc\relax

  \def\ppgccbibauthorsep{,}
  \def\ppgccbibauthorlastsep{,}
  \def\ppgccbibauthorand{and}
  \def\ppgccciteauthorand{and}
  \def\ppgccbibinstring{In}
  \def\ppgccbibpagestring{page}
  \def\ppgccbibpagesstring{pages}
  \def\ppgccbibeditorstring{editor}
  \def\ppgccbibeditorsstring{editors}
  \def\ppgccbibeditionstring{edition}
  \def\ppgccbibchapterstring{chapter}
  \def\ppgccbibtechreportstring{Technical report}
  \def\ppgccbibmscstring{Master's thesis}
  \def\ppgccbibphdstring{PhD thesis}

  \def\ppgcc@appendixname{Appendix}
  \def\ppgcc@attachmentname{Attachment}
}
%    \end{macrocode}
% \subsection*{Folhas de rosto}
%
% \DescribeMacro{\makecoverpage}%
% \DescribeMacro{\maketitlepage}%
% Em geral, o documento possui duas folhas de apresenta?o: a \emph{falsa folha
% de rosto}, que cont? somente o t?ulo da obra; e a \emph{folha de rosto},
% que inclui os nomes do autor e do orientador, a descri?o formal do
% documento, a localidade (cidade e estado) e a data.
%
% Se o documento n? for escrito em portugu?, o comando |\maketitlepage|
% comp?e duas folhas de rosto: uma em portugu? e outra na linguagem do
% documento.
%
%    \begin{macrocode}
\newcommand{\makecoverpage}{%
  \ifppgcc@showcover\begingroup
    \ppgcc@deflanguage{brazil}\ppgcc@selectlanguage
    \ppgcc@maketitlepage{\z@}
  \endgroup\fi
}
\newcommand{\maketitlepage}{%
  \ifppgcc@showtitle
    \ifppgcc@extraporttitlepage\begingroup
      \ppgcc@deflanguage{brazil}\ppgcc@selectlanguage
      \ppgcc@maketitlepage{\@ne}%
    \endgroup\fi
    \ppgcc@maketitlepage{\@ne}%
  \fi
}
%    \end{macrocode}

% \DescribeMacro{\ppgcc@docdescriptionwidth}%
% Esta ?a largura do bloco com a descri?o formal do texto.
%    \begin{macrocode}
\def\ppgcc@docdescriptionwidth{0.5\textwidth}
%    \end{macrocode}

% \DescribeMacro{\ppgcc@makedocdescription}%
% Este comando comp?e o texto com a descri?o formal do documento, que ?parte
% da folha de t?ulo.
%
%    \begin{macrocode}
\def\ppgcc@makedocdescription{%
  \begin{minipage}{\ppgcc@docdescriptionwidth}
    \ppgcc@docdescription
  \end{minipage}%
}
%    \end{macrocode}

% \DescribeMacro{\ppgcc@titpag@fmtauthor}%
% \DescribeMacro{\ppgcc@titpag@fmtadvisor}%
% \DescribeMacro{\ppgcc@titpag@fmtcoadvisor}%
% \DescribeMacro{\ppgcc@titpag@fmttitle}%
% \DescribeMacro{\ppgcc@titpag@fmtaddress}%
% \DescribeMacro{\ppgcc@titpag@fmtdate}%
% Essas macros armazenam a formata?o de cada segmento das folhas de rosto. Se
% for necess?io alterar a formata?o, altere estas macros e \emph{n?} a
% estrutura do comando |\ppgcc@maketitlepage|!
%
%    \begin{macrocode}
\def\ppgcc@titpag@fmtauthor#1{\large\BreakableUppercase{#1}}
\def\ppgcc@titpag@fmtadvisor{\large\scshape}
\let\ppgcc@titpag@fmtcoadvisor\ppgcc@titpag@fmtadvisor
\def\ppgcc@titpag@fmttitle#1{\Large\bfseries\BreakableUppercase{#1}}
\def\ppgcc@titpag@fmtaddress{\large}
\def\ppgcc@titpag@fmtdate{\large}
%    \end{macrocode}

% \DescribeMacro{\ppgcc@maketitlepage}%
% Este comando comp?e as folhas de rosto. ?chamado com um argumento
% obrigat?io, que ?|\z@| para compor a falsa folha de rosto ou |\@ne| para
% compor a folha de rosto.
%
%    \begin{macrocode}

\newenvironment{ppgcctitlingpageany}%
  {%
    \clearpage
    \thispagestyle{titlingpage}%
  }{%
    \thispagestyle{titlingpage}%
    \clearpage
  }

\newenvironment{ppgcctitlingpageodd}%
  {%
    \cleardoublepage
    \thispagestyle{titlingpage}%
  }{%
    \thispagestyle{titlingpage}%
    \clearpage
  }

\def\ppgcc@maketitlepageoptionalpart{%
  \ppgcc@makedocdescription\par
  \vspace{1.5cm}%
  {\ppgcc@titpag@fmtadvisor{\ppgcc@advisor: \ppgcc@advisorname}\par}%
  \ifx\ppgcc@coadvisorname\empty\else
    {\ppgcc@titpag@fmtcoadvisor{\ppgcc@coadvisor: \ppgcc@coadvisorname}\par}%
  \fi
}

\def\ppgcc@maketitlepage#1{%
  \ppgcc@checkauthor
  \begin{ppgcctitlingpageodd}
  \begin{singlespace}
  {\centering
    \vbox to3\baselineskip{%
      \ifx#1\z@
      \else
        {\ppgcc@titpag@fmtauthor{\@author}\par}%
        \vfil
      \fi
    }

    \vspace{\stretch{1}}

    \setbox0=\vbox{\ppgcc@maketitlepageoptionalpart}%
    \edef\ppgcc@textvspace{%
      \noexpand\vrule
      height\the\ht0
      depth\the\dp0
      width0pt\noexpand\par}
    \ppgcc@textvspace

    \vspace{1.5cm}

    \if@selfdoc
      \setbox0=\vbox{%
    \fi
    {\DoubleSpacing\ppgcc@titpag@fmttitle{\@deflang@title}\par}%
    \if@selfdoc
      }%
      \begin{tikzpicture}[overlay,remember picture]
        \node (tltitle) at (0,\ht0) {};
        \node (brtitle) at (\wd0,-\dp0) {};
      \end{tikzpicture}\box0
    \fi

    \vspace{1.5cm}

    \ifx#1\z@
      \ppgcc@textvspace
    \else
      \hspace*{\stretch{1}}%
      \if@selfdoc
        \setbox0=\hbox{%
      \fi
      \ppgcc@makedocdescription
      \if@selfdoc
        }%
        \begin{tikzpicture}[overlay,remember picture]
          \node (tldescr) at (0,\ht0) {};
          \node (brdescr) at (\wd0,-\dp0) {};
          \node (trdescr) at (tldescr -| brdescr) {};
          \node (bldescr) at (tldescr |- brdescr) {};
          %\draw [area] (trdescr) -- (tldescr) -- (bldescr) -- (brdescr);
          \draw [area] (trdescr) -- (tldescr -| left);
          \draw [area] (brdescr) -- (bldescr -| left);
          \draw [area] (tldescr) -- (bldescr);
          \node (bseparrow) at ([xshift=-1cm] tldescr) {};
          \draw [marrow] (bseparrow.center)
            -- node [left] {1,5cm} (bseparrow |- brtitle);
          \draw [marrow] (bldescr) ++(0,-.5cm)
            -- node [below] {Metade da largura}
            ($(brdescr) +(0,-.5cm)$);
          \draw [marrow] (bldescr) ++(-1cm,0)
            -- node [left] {1,5cm} ++(0,-1.5cm);
        \end{tikzpicture}%
        \box0\relax
      \fi

      \vspace{1.5cm}

      {\ppgcc@titpag@fmtadvisor{\ppgcc@advisor: \ppgcc@advisorname}\par}%
      \ifx\ppgcc@coadvisorname\empty\else
        {\ppgcc@titpag@fmtcoadvisor{\ppgcc@coadvisor: \ppgcc@coadvisorname}\par}%
      \fi
    \fi

    \vspace{\stretch{1}}

    \vbox to3\baselineskip{%
      \ifx#1\z@
      \else
        \vfil
        {\ppgcc@titpag@fmtaddress{\@address}\par}%
        \medskip
        {\ppgcc@titpag@fmtdate{\ppgcc@formatdate\par}}%
      \fi
    }\par
  }
  \if@selfdoc\global\let\ppgcc@selfdoc@fg\ppgcc@selfdoc@drawtitlepage\fi
  \end{singlespace}
  \end{ppgcctitlingpageodd}
}
%    \end{macrocode}


% \subsection*{Ficha Catalogr?ica}
%
% A composi?o da Ficha Catalogr?ica ?dividida em alguns assuntos distintos,
% para organizar o c?igo-fonte.
%
%
% \subsubsection*{Ocorr?cia de ilustra?es}
%
% Determinar se h?ou n? ilustra?es no documento ?um pouco mais complicado
% do que parece ?primeira vista. O problema ?que o \LaTeX{} processa as
% informa?es ?medida em que o documento ?processado; no entanto, a ficha
% catalogr?ica aparece muito antes de qualquer ilustra?o. Portanto,
% determinar se a part?ula ``il.'' aparece ou n? na ficha requer duas
% compila?es: uma para varrer a ocorr?cia de comandos ou ambientes que
% incluem ilustra?es (por exemplo, |\begin{figure}|\ldots|\end{figure}|) e
% armazenar o resultado da procura em um arquivo auxiliar (|.aux|); e outra
% para ler essa informa?o logo no in?io do documento para inserir
% condicionalmente a part?ula ``il.''.
%
% \DescribeMacro{\if@fc@ilhookcalled}%
% \DescribeMacro{\if@fc@hasil}%
% Duas condicionais s? definidas atrav? de |\newif| para detectar a
% ocorr?cia de ilustra?es: |\if@fc@ilhookcalled|, que ?ligada quando um
% comando ou ambiente de ilustra?o ?usado no documento, e |\if@fc@hasil|, que
% ?ligada para que a part?ula ``il.'' efetivamente apare? na ficha
% catalogr?ica. A checagem de ilustra?es ?feita em duas etapas (duas
% compila?es): a primeira ?a verifica?o do uso de comandos ou ambientes (que
% chama o comando |\ppgcc@fc@ilhook| e grava uma informa?o no arquivo |.aux|);
% a segunda ?o uso da informa?o do arquivo auxiliar para gerar a ficha
% catalogr?ica (que ocorre antes de qualquer ilustra?o do documento).
%
%    \begin{macrocode}
\newif\if@fc@ilhookcalled\@fc@ilhookcalledfalse
\newif\if@fc@hasil\@fc@hasilfalse
%    \end{macrocode}

% \DescribeMacro{\ppgcc@fc@ilaux}%
% Esee comando ?chamado somente pelo arquivo |.aux|: indica que uma compila?o
% anterior detectou ilustra?es no documento, portanto a part?ula ``il.'' deve
% aparecer na ficha catalogr?ica (ou seja, ativa a condicional
% |\if@fc@hasil|).
%
%    \begin{macrocode}
\def\ppgcc@fc@ilaux{\global\@fc@hasiltrue}
%    \end{macrocode}
%
% \DescribeMacro{\ppgcc@fc@ilhook}%
% Este comando ?chamado por comandos e ambientes que inserem ilustra?es no
% texto. Grava no arquivo |.aux| a informa?o de que a pr?ima compila?o deve
% conter a part?ula ``il.'' na ficha catalogr?ica.
%
%    \begin{macrocode}
\def\ppgcc@fc@ilhook{%
  \if@fc@ilhookcalled\else
    \global\@fc@ilhookcalledtrue
    \global\@fc@hasiltrue
    \immediate\write\@auxout{%
      \string\ppgcc@fc@ilaux{}^^J
    }%
    \typeout{* \ppgcc@classname: Pelo menos uma ilustracao encontrada.}%
  \fi
}
%    \end{macrocode}

% \DescribeMacro{\ppgccaddilcmd}%
% \DescribeMacro{\ppgccaddilenv}%
% Estes comandos permitem especificar comandos e ambientes que correspondem a
% ilustra?es do documento. Ambos possuem um argumento obrigat?io:
% |\ppgccaddilcmd|\marg{\textbackslash comando} ou
% |\ppgccaddilenv|\marg{nome\_do\_ambiente}. \emph{Obs.:} N? ?necess?io
% chamar |\ppgccaddilenv{figure}|.
%
%    \begin{macrocode}
\newcommand{\ppgccaddilcmd}[1]{%
  \begingroup
  \let\ppgcc@temp#1
  \expandafter\gdef\expandafter#1\expandafter{%
    \expandafter\ppgcc@fc@ilhook\ppgcc@temp}%
  \endgroup
}
\newcommand{\ppgccaddilenv}[1]{%
  \expandafter\ppgccaddilcmd\expandafter{\csname#1\endcsname}%
}
%    \end{macrocode}

% O ambiente |figure| ?um dos ambientes de ilustra?o. ?mais seguro definir
% isso no escopo do comando |\AtBeginDocument|, j?que o ambiente |figure| pode
% ser redefinido por algum outro pacote inclu?o pelo usu?io.
%
%    \begin{macrocode}
\AtBeginDocument{\ppgccaddilenv{figure}}
%    \end{macrocode}


% \subsubsection*{Medidas da Ficha Catalogr?ica}
%
% Aqui s? definidas v?ias medidas (dimens?es, espa?mentos, etc.) para a
% montagem da Ficha Catalogr?ica. S? todas definidas como comandos (|\def|),
% pois n? h?necessidade de consumir registros de dimens?es do \TeX{}.
%
% \DescribeMacro{\ppgcc@fc@width}%
% Largura do ret?gulo da ficha (pode-se citar |\textwidth| para adotar a mesma
% largura do bloco de texto):
%
%    \begin{macrocode}
\def\ppgcc@fc@width{12.5cm}
%    \end{macrocode}

% \DescribeMacro{\ppgcc@fc@sep}%
% \DescribeMacro{\ppgcc@fc@rule}%
% Grossura da linha e margem interna do ret?gulo da ficha:
%
%    \begin{macrocode}
\def\ppgcc@fc@sep{1em}
\def\ppgcc@fc@rule{.8pt}
%    \end{macrocode}

% \DescribeMacro{\ppgcc@fc@smallsep}%
% \DescribeMacro{\ppgcc@fc@bigsep}%
% Separa?o vertical entre os par?rafos pr?imos e distantes:
%
%    \begin{macrocode}
\def\ppgcc@fc@smallsep{0cm}
\def\ppgcc@fc@bigsep{1em}
%    \end{macrocode}

% \DescribeMacro{\ppgcc@fc@parindent}%
% \DescribeMacro{\ppgcc@fc@colsep}%
% Indenta?o dos par?rafos e separa?o entre as colunas:
%
%    \begin{macrocode}
\def\ppgcc@fc@parindent{.5cm}
\def\ppgcc@fc@colsep{1em}
%    \end{macrocode}


% \subsubsection*{Montagem da Ficha Catalogr?ica}
%
% Este trecho de c?igo-fonte lida especificamente com os detalhes da montagem
% da ficha catalogr?ica.
%
% Defini?o do contador para armazenar o n?mero de p?inas introdut?ias (a
% informa?o ?colhida automaticamente):
%
%    \begin{macrocode}
\newcounter{ppgcc@fc@frontpages}
%    \end{macrocode}

% Reprograma?o do |\mainmatter| para registrar o n?mero de p?inas
% introdut?ias (desde a ?ltima ocorr?cia de |\frontmatter|), para constar na
% Ficha Catalogr?ica:
%
%    \begin{macrocode}
\let\ppgcc@mainmatter=\mainmatter
\def\mainmatter{
  \clearpage
  \setcounter{ppgcc@fc@frontpages}{\c@page}%
  \addtocounter{ppgcc@fc@frontpages}{-1}%
  \immediate\write\@auxout{%
    \string\setcounter{ppgcc@fc@frontpages}{\the\c@ppgcc@fc@frontpages}^^J
  }%
  \ppgcc@mainmatter
}
%    \end{macrocode}

% Comandos auxiliares para checar defini?es obrigat?ias e disparar erros se
% as defini?es n? tiverem sido feitas:
%
%    \begin{macrocode}
\def\ppgcc@fc@checkerr#1#2#3{%
  \ClassError{\ppgcc@classname}{%
    Para a montagem da Ficha Catalografica e' necessario\MessageBreak
    definir #1 (chave `#2'\MessageBreak
    ou comando \noexpand#3)}{\@ehd}
    \endinput
}
\def\ppgcc@fc@checkwarn#1#2#3{%
  \ClassWarning{\ppgcc@classname}{%
    Para a montagem da Ficha Catalografica e' necessario\MessageBreak
    definir #1 (chave `#2'\MessageBreak
    ou comando \noexpand#3)}
}
\def\ppgcc@fc@checkparamerr#1#2#3#4{%
  \@ifundefined{#1}
    {\ppgcc@fc@checkerr{#2}{#3}{#4}}{}
}
\def\ppgcc@fc@checkparamwarn#1#2#3#4{%
  \@ifundefined{#1}
    {\ppgcc@fc@checkwarn{#2}{#3}{#4}}{}
}
\def\ppgcc@fc@check{%
  \ppgcc@fc@checkparamerr{ppgcc@authorrev}{o nome reverso do autor}{authorrev}
    {\authorrev}
  \ppgcc@fc@checkparamwarn{ppgcc@cutter}{o c?igo ``cutter''}{cutter}{\cutter}
  \ppgcc@fc@checkparamwarn{ppgcc@cdu}{o c?igo CDU}{cdu}{\cdu}
}
%    \end{macrocode}

% Comando para converter as quebras de linha (|\\|) do t?ulo em espa?s:
%
%    \begin{macrocode}
\def\ppgcc@titlenobreak{%
  \begingroup
    \let\\=\space
    \@title
  \endgroup
}
%    \end{macrocode}

% Defini?o e c?culo de algumas dimens?es auxiliares para a composi?o da
% ficha:
%
%    \begin{macrocode}
\newlength{\ppgcc@l@fc@width}
\newlength{\ppgcc@l@fc@seccolwidth}
\newlength{\ppgcc@l@fc@firstcolwidth}

\def\ppgcc@fc@calc{%
  \ppgcc@fc@check

  \@ifundefined{ppgcc@cutter}{%
    \setlength{\ppgcc@l@fc@firstcolwidth}{0pt}
    \def\ppgcc@fc@realcolsep{0pt}
  }{%
    \settowidth{\ppgcc@l@fc@firstcolwidth}{\ppgcc@cutter}
    \let\ppgcc@fc@realcolsep=\ppgcc@fc@colsep
  }

  \setlength{\ppgcc@l@fc@width}{\ppgcc@fc@width}
  \addtolength{\ppgcc@l@fc@width}{-\ppgcc@fc@rule}
  \addtolength{\ppgcc@l@fc@width}{-\ppgcc@fc@rule}
  \addtolength{\ppgcc@l@fc@width}{-\ppgcc@fc@sep}
  \addtolength{\ppgcc@l@fc@width}{-\ppgcc@fc@sep}

  \setlength{\ppgcc@l@fc@seccolwidth}{\ppgcc@l@fc@width}
  \addtolength{\ppgcc@l@fc@seccolwidth}{-\ppgcc@l@fc@firstcolwidth}
  \addtolength{\ppgcc@l@fc@seccolwidth}{-\ppgcc@fc@realcolsep}
}
%    \end{macrocode}

% \DescribeMacro{\ppgcc@makefc}%
% Comando que comp?e a p?ina com a Ficha Catalogr?ica:
%
%    \begin{macrocode}
\def\ppgcc@makefc{%
  \begin{ppgcctitlingpageany}%
  \begin{singlespace}%

  \ppgcc@checkauthor
  \ppgcc@fc@calc

  \hyphenpenalty=10000
  \flushbottom

  \noindent
  \begin{tabular}{@{}ll@{}}
    \copyright & \ppgcc@year, \@author. \\
    & Todos os direitos reservados.
  \end{tabular}

  \leavevmode\vfill

  \fboxsep=\ppgcc@fc@sep
  \fboxrule=\ppgcc@fc@rule
  \centering\noindent
  \if@selfdoc
    \setbox0=\hbox{%
  \fi
  \fbox{%
  \begin{minipage}{\ppgcc@l@fc@width}
    \begin{tabular}%
      {@{}p{\ppgcc@l@fc@firstcolwidth}%
      @{\hspace{\ppgcc@fc@realcolsep}}%
      >{\raggedright\arraybackslash}%
      p{\ppgcc@l@fc@seccolwidth}@{}}
      & \ppgcc@authorrev
        \\[\ppgcc@fc@bigsep]
      \@ifundefined{ppgcc@cutter}{}{\ppgcc@cutter}%
      & \hspace*{\ppgcc@fc@parindent}%
        \ppgcc@titlenobreak{} / \@author.\space --- \@address, \ppgcc@year
        \\[\ppgcc@fc@smallsep]
      & \hspace*{\ppgcc@fc@parindent}%
        \roman{ppgcc@fc@frontpages}, \arabic{lastpage} f.~%
        \if@fc@hasil : il.~\fi
        ; 29cm
        \\[\ppgcc@fc@bigsep]
      & \hspace*{\ppgcc@fc@parindent}%
        \ifppgcc@phd
          Tese (doutorado)
        \else
          Disserta\c{c}\~{a}o (mestrado)
        \fi
        --- \ppgcc@fcuniversity
        \\[\ppgcc@fc@smallsep]
      & \hspace*{\ppgcc@fc@parindent}%
        Orientador: \ppgcc@advisorname
        \\[\ppgcc@fc@bigsep]
      & \hspace*{\ppgcc@fc@parindent}%
        \ppgcc@indexkeys
    \end{tabular}
    \@ifundefined{ppgcc@cdu}{}{%
      \vspace{\ppgcc@fc@bigsep}\par
      {\centering
        CDU \ppgcc@cdu
      \par}%
    }%
  \end{minipage}}%
  \if@selfdoc
    }\begingroup
      \edef\@ht{\the\ht0}\edef\@dp{\the\dp0}%
      \box0\relax
      \ppgcc@selfdoc@begin
      \ppgcc@selfdoc@drawfc
      \ppgcc@selfdoc@end
    \endgroup
  \fi
  \end{singlespace}
  \end{ppgcctitlingpageany}
  \clearpage
}
%    \end{macrocode}

% \DescribeMacro{\makefc}%
% Finalmente, o comando de usu?io para disparar condicionalmente a composi?o
% da p?ina com a Ficha Catalogr?ica:
%
%    \begin{macrocode}
\newcommand{\makefc}{%
  \ifppgcc@showfc\ppgcc@makefc\fi
}
%    \end{macrocode}


% \subsection*{A Folha de Aprova?o}
%
% A Folha de Aprova?o ?fornecida pela Secretaria do Curso. Ela deve ser
% digitalizada para a inclus? na vers? final do documento. Cheque a op?o
% |approval| (ou o comando |\approval|) para maiores informa?es.
%
% \DescribeMacro{\makeapproval}%
% Esta macro comp?e a Folha de Aprova?o. Se a op?o |approval| (ou o comando
% |\approval|) tiver sido usada, o documento digitalizado ser?inclu?o; caso
% contr?io, ser?gerada uma p?ina com as instru?es para digitalizar a Folha
% de Aprova?o.
%
%    \begin{macrocode}
\newcommand{\makeapproval}{%
  \ifppgcc@showapproval
    \begin{ppgcctitlingpageodd}
    \begin{singlespace}

    \ifppgcc@hasapprovalpages
      \def\ppgcc@typesetapprovalpage{%
        \thispagestyle{titlingpage}%
        \setlength{\TPHorizModule}{\ppgcc@approvalwidth}%
        \begin{textblock}{1}(0,0)%
          \vspace*{\ppgcc@approvalvertadjust}%
          \noindent\includegraphics[width=\ppgcc@approvalwidth]{\ppgcc@approval}\par
        \end{textblock}%
        \clearpage
        }%
      \ppgcc@doapprovallist{\ppgcc@typesetapprovalpage}%
    \else
      \leavevmode\vfil
      {\centering
        {\huge [Folha de Aprova\c{c}\~ao]\par}
        \bigskip
        Quando a secretaria do Curso fornecer esta folha,\par
        ela deve ser digitalizada e armazenada no disco em formato
          gr\'afico.\par
        \medskip
        Se voc\^e estiver usando o \texttt{pdflatex},\par
        armazene o arquivo preferencialmente em formato PNG\par
        (o formato JPEG \'e pior neste caso).\par
        \medskip
        Se voc\^e estiver usando o \texttt{latex} (n\~ao o
          \texttt{pdflatex}),\par
        ter\'a que converter o arquivo gr\'afico para o formato EPS.\par
        \medskip
        Em seguida, acrescente a op\c{c}\~ao
        \texttt{approval=\{}\emph{nome do arquivo}\texttt{\}}\par
        ao comando \texttt{\textbackslash ppgccufmg}.\par
        \medskip
        Se a imagem da folha de aprova\c{c}\~ao precisar ser ajustada, use:\par
        \texttt{approval=[}\emph{ajuste}\texttt{][}\emph{escala}\texttt{]\{}%
        \emph{nome do arquivo}\texttt{\}}\par
        onde \emph{ajuste} ?uma dist\^ancia para deslocar a imagem para baixo\par
        e \emph{escala} \'e um fator de escala para a imagem. Por exemplo:\par
        \texttt{approval=[-2cm][0.9]\{}\emph{nome do arquivo}\texttt{\}}\par
        desloca a imagem 2cm para cima e a escala em 90\%.\par
      }
    \fi

    \end{singlespace}
    \end{ppgcctitlingpageodd}
  \fi
}
%    \end{macrocode}


% \subsection*{Resumos e \emph{Abstracts}}
%
% O trabalho pode incluir uma ou mais vers?es de resumos (\emph{abstracts}):
% tradicionalmente h?uma vers? em portugu? e outra em ingl?, mas podem ser
% exigidas duas vers?es em uma mesma l?gua (``Resumo'' e ``Resumo Estendido'',
% por exemplo).
%
% Para comportar a variedade de possibilidades, a classe \cls{ppgccufmg}
% permite a defini?o de um n?mero arbitr?io de resumos. Para cada um,
% especifica-se o t?ulo (``Resumo'', ``Abstract'', etc.) e um arquivo |.tex|
% com o texto do resumo em quest?.
%
% O arquivo |.tex| especificado pode vir com a especifica?o de um diret?io,
% mas cuidado: os compiladores \LaTeX{} n? gostam muito de nomes de arquivos
% (ou diret?ios) com espa?s ou letras acentuadas. Portanto, n? use nomes de
% arquivo como ``\texttt{Cap?ulo 1.tex}''. Al? disso, ao especificar
% diret?ios use barras (|/|) ao inv? de contra-barras (|\|), mesmo que voc?% esteja trabalhando no Microsoft Windows.
%
% Nenhum desses arquivos |.tex| especificados deve conter o comando |\chapter|.
% Ele ?chamado automaticamente. Al? disso, se for necess?io usar |\section|,
% |\subsection|, etc., coloque um asterisco depois do comando (ou seja,
% |\section*|, |\subsection*|, etc.), para que as se?es n? apare?m no
% Conte?do.
%
% \DescribeMacro{\includeabstract}%
% Este comando inclui uma p?ina de resumo. A forma de uso ?
% |\includeabstract| \marg{t?ulo} \marg{arquivo}. \meta{t?ulo} ?o t?ulo do
% resumo, conforme explicado anteriormente.
%
%    \begin{macrocode}
\newcommand{\includeabstract}[3][]{%
  \ifppgcc@showabstract
    \cleardoublepage
    \begingroup
      \def\ppgcc@temp{#1}%
      \ifx\ppgcc@temp\empty\else
        \ppgcc@deflanguage{#1}\ppgcc@selectlanguage
      \fi
      \ppgcc@redefchaptitlefont
      \chapter{#2}
      \input{#3}
    \endgroup
    \clearpage
  \fi
}
%    \end{macrocode}

% Para a inclus? de v?ios arquivos de resumo, ?necess?io um contador para
% armazenar o n?mero de resumos definidos:
%
%    \begin{macrocode}
\newcounter{ppgcc@c@abstract}
%    \end{macrocode}

% \DescribeMacro{\addabstract}%
% Agora o comando que registra um arquivo de resumo: |\addabstract|
% \marg{t?ulo} \marg{arquivo}.
%
%    \begin{macrocode}
\newcommand{\addabstract}[3][]{%
  \stepcounter{ppgcc@c@abstract}%
  \expandafter\gdef
    \csname ppgcc@abstract@\theppgcc@c@abstract lang\endcsname{#1}%
  \expandafter\gdef
    \csname ppgcc@abstract@\theppgcc@c@abstract title\endcsname{#2}%
  \expandafter\gdef
    \csname ppgcc@abstract@\theppgcc@c@abstract filename\endcsname{#3}%
}
%    \end{macrocode}

% \DescribeMacro{\ppgcc@includeabstractlist}%
% Este comando inclui todos os resumos previamente definidos pelo comando
% |\addabstract|.
%
%    \begin{macrocode}
\newcommand{\ppgcc@includeabstractlist}{%
  \ifppgcc@showabstract
    \setcounter{ppgcc@c@abstract}{1}
    \loop\expandafter\ifx
      \csname ppgcc@abstract@\theppgcc@c@abstract title\endcsname
      \relax
      \else
      \includeabstract
        [\csname ppgcc@abstract@\theppgcc@c@abstract lang\endcsname]
        {\csname ppgcc@abstract@\theppgcc@c@abstract title\endcsname}
        {\csname ppgcc@abstract@\theppgcc@c@abstract filename\endcsname}
      \stepcounter{ppgcc@c@abstract}
    \repeat
  \fi
}
%    \end{macrocode}

% \DescribeMacro{\keywords}%
% Este comando deve ser usado no fim dos arquivos de resumo para especificar as
% palavras-chave do documento. A forma de uso ? \cmd{\includeabstract}
% \marg{chave, chave, \ldots}.
%
%    \begin{macrocode}
\newcommand{\keywords}[1]{%
  \bigskip
  \noindent\textbf{\ppgcc@keywordsname:} #1.%
}
%    \end{macrocode}


% \subsection*{Dedicat?ia e agradecimentos}
%
% As p?inas de dedicat?ia e agradecimentos s? baseadas em arquivos |.tex|
% externos, j?que n? h?maneira de comp?las automaticamente :-) . Todos os
% comandos recebem como argumento obrigat?io o nome do arquivo |.tex| (que
% pode incluir o diret?io), mas valem as mesmas regras descritas para os
% resumos: nomes de arquivos n? podem ter espa?s ou letras acentuadas; sempre
% especifique diret?ios com barras (|/|) ao inv? de contra-barras (|\|); n?
% use o comando |\chapter|; e coloque asterisco nos comandos |\section*|,
% |\subsection*|, etc.
%
% Embora o mais comum seja a inclus? de somente uma p?ina de cada
% (dedicat?ia e agradecimentos), h?autores que preferem fornecer mais de uma
% vers? de cada (em portugu? e em l?gua estrangeira). Portanto, ?poss?el
% citar v?ias dedicat?ias e/ou agradecimentos pelo uso repetido dos comandos
% |\includededication| e |\includeack| (ou das op?es |dedication| e |ack| do
% comando |\ppgccufmg|).
%
% Para a inclus? de v?ios arquivos de cada, ?necess?io um contador para
% armazenar o n?mero de arquivos definidos:
%
%    \begin{macrocode}
\newcounter{ppgcc@c@dedication}
\newcounter{ppgcc@c@ack}
%    \end{macrocode}

% \DescribeMacro{\adddedication}%
% \ldots e o comando que registra um arquivo de agradecimentos:
% |\adddedication| \marg{arquivo} (note que a dedicat?ia n? requer um
% t?ulo).
%
%    \begin{macrocode}
\newcommand{\adddedication}[1]{%
  \stepcounter{ppgcc@c@dedication}
  \expandafter\def\csname ppgcc@dedication@\theppgcc@c@dedication
    filename\endcsname{#1}
}
%    \end{macrocode}

% \DescribeMacro{\addack}%
% Agora o comando que registra um arquivo de agradecimentos: |\addack|
% \oarg{t?ulo} \marg{arquivo}. Se \meta{t?ulo} for omitido, ser?adotado
% ``Agradecimentos'' (ou a tradu?o correspondente ?linguagem configurada).
%
%    \begin{macrocode}
\newcommand{\addack}[2][]{%
  \stepcounter{ppgcc@c@ack}
  \def\ppgcc@tempack{#1}
  \ifx\ppgcc@tempack\empty
    \expandafter\addack@\expandafter{\ppgcc@acknowledgments}
  \else
    \addack@{#1}
  \fi
  \expandafter\def\csname ppgcc@ack@\theppgcc@c@ack filename\endcsname{#2}
}
\def\addack@#1{
  \expandafter\def\csname ppgcc@ack@\theppgcc@c@ack title\endcsname{#1}
}
%    \end{macrocode}

% \DescribeMacro{\includededication}%
% Este comando inclui a dedicat?ia. A forma de uso ? |\includededication|
% \marg{arquivo}.
%
%    \begin{macrocode}
\newcommand{\includededication}[1]{%
  \ifppgcc@showdedication
    \cleardoublepage
    \vspace*{\stretch{1}}
    \textit{\input{#1}}
    \vspace*{\stretch{1}}
    \clearpage
  \fi
}
%    \end{macrocode}

% \DescribeMacro{\includeack}%
% Este comando inclui os agradecimentos. A forma de uso ? |\includeack|
% \oarg{t?ulo} \marg{arquivo}.
%
%    \begin{macrocode}
\newcommand{\includeack}[2][]{%
  \ifppgcc@showack
    \edef\ppgcc@tempack{#1}%
    \ifx\ppgcc@tempack\empty\let\ppgcc@tempack\ppgcc@acknowledgments\fi
    \ppgcc@redefchaptitlefont
    \chapter{\ppgcc@tempack}
    \input{#2}
    \clearpage
  \fi
}
%    \end{macrocode}

% \DescribeMacro{\ppgcc@includededicationlist}%
% Este comando inclui todos os resumos previamente definidos pelo comando
% |\adddedication|.
%
%    \begin{macrocode}
\newcommand{\ppgcc@includededicationlist}{%
  \ifppgcc@showdedication
    \setcounter{ppgcc@c@dedication}{1}
    \loop
    \expandafter\ifx
      \csname ppgcc@dedication@\theppgcc@c@dedication filename\endcsname
      \relax
    \else
      \includededication
        {\csname ppgcc@dedication@\theppgcc@c@dedication filename\endcsname}
      \stepcounter{ppgcc@c@dedication}
    \repeat
  \fi
}
%    \end{macrocode}
% \DescribeMacro{\ppgcc@includeacklist}%
% Este comando inclui todos os resumos previamente definidos pelo comando
% |\addack|.
%
%    \begin{macrocode}
\newcommand{\ppgcc@includeacklist}{%
  \ifppgcc@showack
    \setcounter{ppgcc@c@ack}{1}
    \loop
    \expandafter\ifx\csname ppgcc@ack@\theppgcc@c@ack filename\endcsname\relax
    \else
      \includeack
        [\csname ppgcc@ack@\theppgcc@c@ack title\endcsname]
        {\csname ppgcc@ack@\theppgcc@c@ack filename\endcsname}
      \stepcounter{ppgcc@c@ack}
    \repeat
  \fi
}
%    \end{macrocode}


% \subsection*{Ep?rafe}
%
% \DescribeMacro{\ppgcc@epigraph@fmttext}%
% \DescribeMacro{\ppgcc@epigraph@fmtauthor}%
% Essas macros armazenam a formata?o de cada segmento da ep?rafe. Se for
% necess?io alterar a formata?o, altere estas macros e \emph{n?} a
% estrutura do comando |\makeepigraph|!
%
%    \begin{macrocode}
\def\ppgcc@epigraph@fmttext#1{\textit{``#1''}}
\def\ppgcc@epigraph@fmtauthor#1{(#1)}
%    \end{macrocode}
%
% \DescribeMacro{\ppgcc@epigraph@fmtvspace}%
% O espa? vertical entre o texto e o autor da ep?rafe ?definido pela macro
% |\ppgcc@epigraph@fmtvspace|. Note que esta ?apenas uma macro, n? uma
% dimens?.
%
%    \begin{macrocode}
\def\ppgcc@epigraph@fmtvspace{0em}
%    \end{macrocode}

% \DescribeMacro{\makeepigraph}%
% Comp?e a p?ina com o texto da ep?rafe. A p?ina ?composta somente se a
% op?o |epigraphtext| ou o comando |\epigraphtext| forem usados.
%
%    \begin{macrocode}
\newcommand{\makeepigraph}{%
  \ifppgcc@showepigraph
    \@ifundefined{ppgcc@epigraphtext}{}{%
      \cleardoublepage
      \vspace*{\stretch{1}}
      \begingroup
        \raggedleft
        {\ppgcc@epigraph@fmttext{\ppgcc@epigraphtext}\par}%
        \vspace{\ppgcc@epigraph@fmtvspace}
        {\ppgcc@epigraph@fmtauthor{\ppgcc@epigraphauthor}\par}%
      \endgroup
      \clearpage
    }%
  \fi
}
%    \end{macrocode}


% \subsection*{Cabe?lhos e rodap?}
%
% O c?igo a seguir lida com todo o necess?io para configurar os cabe?lhos e
% rodap? das p?inas.
%
%    \begin{macrocode}
\if@twoside
  \makeheadrule{headings}{\textwidth}{0pt}
  \makeoddhead{headings}{\textsc{\rightmark}}{}{\thepage}
  \makeevenhead{headings}{\thepage}{}{\textsc{\leftmark}}

  \copypagestyle{contents}{plain}

  \copypagestyle{listoffigures}{plain}

  \copypagestyle{listoftables}{plain}

  \copypagestyle{bibliography}{headings}
  \makeoddhead{bibliography}{\textsc{\bibname}}{}{\thepage}
  \makeevenhead{bibliography}{\thepage}{}{\textsc{\bibname}}
\else
  \makeheadrule{headings}{\textwidth}{0pt}
  \makeoddhead{headings}{\textsc{\leftmark}}{}{\thepage}

  \copypagestyle{contents}{plain}

  \copypagestyle{listoffigures}{plain}

  \copypagestyle{listoftables}{plain}

  \copypagestyle{bibliography}{headings}
  \makeoddhead{bibliography}{\textsc{\bibname}}{}{\thepage}
\fi

\def\ppgcc@chaptermark#1{%
  \markboth{%
    \ifnum\c@secnumdepth>\m@ne
      \if@mainmatter
        \if@twoside\@chapapp\ \fi
        \thechapter.
      \fi
    \fi
    #1}{}}%

\def\ppgcc@sectionmark#1{%
  \markright{%
    \ifnum\c@secnumdepth>\z@\thesection. \ \fi
    #1}}%

\def\ppgccrestoremarks{%
  \let\chaptermark=\ppgcc@chaptermark
  \let\sectionmark=\ppgcc@sectionmark
}

\ppgccrestoremarks
\g@addto@macro\mainmatter{\pagestyle{headings}\ppgccrestoremarks}
%    \end{macrocode}


% \subsection*{Sum?io, Lista de Figuras e Lista de Tabelas}
%
% \DescribeMacro{\ppgcclistoffigures}%
% \DescribeMacro{\ppgcclistoftables}%
% \DescribeMacro{\ppgcctableofcontents}%
% Estes comandos comp?em a Lista de Figuras, a Lista de Tabelas e o Sum?io. Os
% comandos originais s? modificados para lidar corretamente com a formata?o
% dos cabe?lhos e rodap? das p?inas.
%
%    \begin{macrocode}
\let\ppgcc@orig@listoffigures=\listoffigures
\renewcommand{\listoffigures}{{%
  \ifppgcc@showlof\begingroup
    \cleardoublepage
    \ppgcc@redefchaptitlefont
    \let\ppgcc@save@listoffigures=\listoffigures
    \let\listoffigures=\ppgcc@orig@listoffigures
    \pagestyle{listoffigures}
    \listoffigures
    \clearpage
  \endgroup\fi
}}

\let\ppgcc@orig@listoftables=\listoftables
\renewcommand{\listoftables}{{%
  \ifppgcc@showlot\begingroup
    \cleardoublepage
    \ppgcc@redefchaptitlefont
    \let\ppgcc@save@listoftables=\listoftables
    \let\listoftables=\ppgcc@orig@listoftables
    \pagestyle{listoftables}
    \listoftables
    \clearpage
  \endgroup\fi
}}

\let\ppgcc@orig@tableofcontents=\tableofcontents
\renewcommand{\tableofcontents}{{%
  \ifppgcc@showtoc\begingroup
    \cleardoublepage
    \ppgcc@redefchaptitlefont
    \let\ppgcc@save@tableofcontents=\tableofcontents
    \let\tableofcontents=\ppgcc@orig@tableofcontents
    \pagestyle{contents}
    \tableofcontents*
    \clearpage
  \endgroup\fi
}}
%    \end{macrocode}


% \subsection*{O comando ``deixe que eu fa? tudo''}
%
% \DescribeMacro{\ppgccufmg}%
% Este ?o ``comando m?ico'' que comp?e todas as p?inas introdut?ias. Ele
% recebe como argumento uma s?ie de pares do tipo \meta{chave}|=|\meta{valor},
% descritos na Se?o~\ref{sec:ppgccufmg}. Leia essa se?o!
%
%    \begin{macrocode}

\let\ppgcc@beforetoc\empty

\newcommand\ppgcc@addabstract[3][]{\addabstract[#1]{#2}{#3}}
\def\ppgcc@adddedication#1{\adddedication{#1}}
\def\ppgcc@addack{\@ifnextchar[{\addack}{\ppgcc@addack@}}
\def\ppgcc@addack@#1\relax{\addack[]{#1}}
\def\ppgcc@addbeforetoc#1{\g@addto@macro\ppgcc@beforetoc{#1}}

\define@key{ppgcc}{title}{\title{#1}}
\define@key{ppgcc}{author}{\author{#1}}
\define@key{ppgcc}{authorrev}{\authorrev{#1}}
\define@key{ppgcc}{cutter}{\cutter{#1}}
\define@key{ppgcc}{cdu}{\cdu{#1}}
\define@key{ppgcc}{indexkeys}{\indexkeys{#1}}
\define@key{ppgcc}{university}{\university{#1}}
\define@key{ppgcc}{fcuniversity}{\fcuniversity{#1}}
\define@key{ppgcc}{course}{\course{#1}}
\define@key{ppgcc}{portuguesetitle}{\portuguesetitle{#1}}
\define@key{ppgcc}{portugueseuniversity}{\portugueseuniversity{#1}}
\define@key{ppgcc}{portuguesecourse}{\portuguesecourse{#1}}
\define@key{ppgcc}{keywords}{\fckeywords{#1}}
\define@key{ppgcc}{postkeywords}{\fcpostkeywords{#1}}
\define@key{ppgcc}{address}{\address{#1}}
\define@key{ppgcc}{date}{\date{#1}}
\define@key{ppgcc}{approval}{\ppgcc@addapproval{#1}}
\define@key{ppgcc}{epigraphtext}{\epigraphtext#1}
\define@key{ppgcc}{advisor}{\advisor{#1}}
\define@key{ppgcc}{coadvisor}{\coadvisor{#1}}
\define@key{ppgcc}{abstract}{\ppgcc@addabstract#1}
\define@key{ppgcc}{dedication}{\ppgcc@adddedication{#1}}
\define@key{ppgcc}{ack}{\ppgcc@addack#1\relax}
\define@key{ppgcc}{beforetoc}{\ppgcc@addbeforetoc{#1}}

\fcpostkeywords{T?ulo}

\newcommand{\ppgccufmg}[1]{%
  \setkeys{ppgcc}{#1}

  \pagestyle{plain}

  \frontmatter

  % Falsa folha de rosto
  \makecoverpage
  % Folha(s) de rosto
  \maketitlepage
  % Ficha Catalogr?ica
  \makefc
  % Folha de aprova?o
  \makeapproval
  % Dedicat?ia
  \ppgcc@includededicationlist
  % Agradecimentos
  \ppgcc@includeacklist
  % Ep?rafe
  \makeepigraph
  % Resumo, Abstract, etc.
  \ppgcc@includeabstractlist

  % Sum?io e lista de figuras e de tabelas
  \listoffigures
  \listoftables
  \ppgcc@beforetoc
  \tableofcontents

  % ***** ATEN?O *****
  %
  % Se precisar incluir outras listas, como a Lista de Algoritmos ou de
  % S?bolos, N? ALTERE ESTE ARQUIVO PARA COLOCAR O COMANDO AQUI, sen? o
  % documento ser?gerado com problemas! Veja o exemplo da se?o "Para os
  % impacientes", que descreve esse caso.

  \ifppgcc@putmainmatter\mainmatter\fi
}
%    \end{macrocode}

%
% \subsection*{Refer?cias bibliogr?icas}
%
% Aqui est? todos os comandos, op?es e processamentos relacionados ?% composi?o das refer?cias bibliogr?icas (caso se queira adotar o estilo
% \bst{ppgccufmg} de bibliografia para o Bib\TeX).
%
% A seguir, as macros que definem as op?es configur?eis:
%
%    \begin{macrocode}
\def\ppgcc@bibauthorformat{}
\def\ppgcc@bibtitleformat{}
\def\ppgcc@bibbtitleformat{\itshape}
\def\ppgcc@bibbooktitleformat{\itshape}
\def\ppgcc@bibjournalformat{\itshape}

\def\ppgcc@norepstring{---------}
%    \end{macrocode}

% As macros a seguir devem ser usadas de maneira autom?ica e somente dentro do
% arquivo |.bbl| file (que ?gerado pelo sistema Bib\TeX):
%
%    \begin{macrocode}
\newcommand{\ppgccbibauthordoformat}[1]{%
  \if@bibnorepauthor
    \def\ppgcc@currentauthor{#1}%
    \ifx\ppgcc@lastauthor\ppgcc@currentauthor\ppgcc@norepstring
      \else{\ppgcc@bibauthorformat#1}\fi
    \def\ppgcc@lastauthor{#1}%
  \else
    {\ppgcc@bibauthorformat#1}%
  \fi
}
\newcommand{\ppgccbibtitledoformat}[1]{{\ppgcc@bibtitleformat#1}}
\newcommand{\ppgccbibbtitledoformat}[1]{{\ppgcc@bibbtitleformat#1}}
\newcommand{\ppgccbibbooktitledoformat}[1]{{\ppgcc@bibbooktitleformat#1}}
\newcommand{\ppgccbibjournaldoformat}[1]{{\ppgcc@bibjournalformat#1}}
\let\ppgccyeardoformat\empty
\let\ppgccpagedoformat\empty
\let\ppgccnumberdoformat\empty
\let\ppgccvolumedoformat\empty
\let\ppgccisbndoformat\empty
\let\ppgccissndoformat\empty
%    \end{macrocode}

% Eis a declara?o das op?es para o comando |\ppgccbibliography|:
%
%    \begin{macrocode}
\define@key{ufmgbib}{noauthorrepeat}[true]{\@bibnorepauthortrue}
\define@key{ufmgbib}{noauthorrepstring}{\def\ppgcc@norepstring{#1}}
\define@key{ufmgbib}{nobreakitems}[true]{\@nobreakitemstrue}
\define@key{ufmgbib}{bibauthorand}{\def\ppgccbibauthorand{#1}}
\define@key{ufmgbib}{citeauthorand}{\def\ppgccciteauthorand{#1}}
\define@key{ufmgbib}{authorformat}{\def\ppgcc@bibauthorformat{#1}}
\define@key{ufmgbib}{titleformat}{\def\ppgcc@bibtitleformat{#1}}
\define@key{ufmgbib}{btitleformat}{\def\ppgcc@bibbtitleformat{#1}}
\define@key{ufmgbib}{booktitleformat}{\def\ppgcc@bibbooktitleformat{#1}}
\define@key{ufmgbib}{journalformat}{\def\ppgcc@bibjournalformat{#1}}
%    \end{macrocode}

% \DescribeMacro{\ppgccbibliography}%
% Finalmente, eis o comando m?ico. O argumento obrigat?io ?o nome do arquivo
% de banco de dados bibliogr?ico (o arquivo |.bib|), sem a extens?.
%
%    \begin{macrocode}
\newcommand{\ppgccbibliography}[2][]{%
  \setkeys{ufmgbib}{#1}

  \cleardoublepage
  \pagestyle{bibliography}
  \bibliographystyle{ppgccufmg}
  \begingroup
    \ppgcc@redefchaptitlefont
    \if@nobreakitems\raggedbottom\interlinepenalty=10000\relax\fi
    \bibliography{#2}
  \endgroup

  \clearpage
  \pagestyle{headings}
  \ppgccrestoremarks
}
%    \end{macrocode}

%
% \subsection*{Ap?dices e anexos}
%
% \DescribeEnv{appendices}%
% \DescribeEnv{attachments}%
% Esses comandos auxiliam a cria?o de ap?dices e anexos, configurando a
% numera?o em seq??cia ar?ica mai?scula (A, B, etc.) e acrescentando o
% indicador ``Ap?dice'' ou ``Anexo'' no Sum?io.
%
%    \begin{macrocode}
\def\ppgcc@toc@switchto#1{%
  \ifx\ppgcc@chapternumberline\undefined
    \let\ppgcc@chapternumberline\chapternumberline
  \fi
  \def\chapternumberline{#1\space\ppgcc@chapternumberline}%
}
\def\ppgcc@toc@recover{%
  \let\chapternumberline\ppgcc@chapternumberline
  \let\ppgcc@chapternumberline\undefined
}

\let\ppgcc@appendices\appendices\let\appendices\undefined
\let\ppgcc@endappendices\endappendices\let\endappendices\undefined
\newenvironment{appendices}{%
    \cleardoublepage
    \ppgcc@appendices
    \appendix
    \let\appendixname\ppgcc@appendixname
    \immediate\write\@auxout{%
      \string\@writefile{toc}{%
      \string\ppgcc@toc@switchto\string\ppgcc@appendixname}}%
  }{%
    \ppgcc@endappendices
    \immediate\write\@auxout{%
      \string\@writefile{toc}{\string\ppgcc@toc@recover}}
  }

\newenvironment{attachments}{%
    \cleardoublepage
    \ppgcc@appendices
    \appendix
    \let\appendixname\ppgcc@attachmentname
    \def\Hy@appendixstring{attachment}%
    \immediate\write\@auxout{%
      \string\@writefile{toc}{%
      \string\ppgcc@toc@switchto\string\ppgcc@attachmentname}}%
    % Para o hyperref:
    \def\Hy@chapapp{attachment}%
  }{%
    \ppgcc@endappendices
    \immediate\write\@auxout{%
      \string\@writefile{toc}{\string\ppgcc@toc@recover}}
  }
%    \end{macrocode}


% \subsection*{Comandos de formata?o padr?}
%
% O comando |\sloppy| ?disparado para for?r a obedi?cia ?margem direita do
% texto.
%
%    \begin{macrocode}
\sloppy
%    \end{macrocode}

% Defini?o do recuo padr? de par?rafo.
%
%    \begin{macrocode}
\setlength{\parindent}{1cm}
%    \end{macrocode}

% Defini?o dos estilos de t?ulos de cap?ulos, se?es, subse?es, etc.

%    \begin{macrocode}
\makechapterstyle{ppgccufmgsf}{%
  \chapterstyle{default}%
  \expandafter\def\expandafter\chapnumfont\expandafter{%
    \chapnumfont\sffamily}%
  \expandafter\def\expandafter\chapnamefont\expandafter{%
    \chapnamefont\sffamily}%
  \expandafter\def\expandafter\chaptitlefont\expandafter{%
    \chaptitlefont\sffamily}%
  \expandafter\def\expandafter\secheadstyle\expandafter{%
    \secheadstyle\sffamily}%
  \expandafter\def\expandafter\subsecheadstyle\expandafter{%
    \subsecheadstyle\sffamily}%
  \expandafter\def\expandafter\subsubsecheadstyle\expandafter{%
    \subsubsecheadstyle\sffamily}%
  \expandafter\def\expandafter\paraheadstyle\expandafter{%
    \paraheadstyle\sffamily}%
  \expandafter\def\expandafter\subparaheadstyle\expandafter{%
    \subparaheadstyle\sffamily}%
  \setlength{\beforechapskip}{20mm}%
  \setlength{\midchapskip}{7mm}%
  \setlength{\afterchapskip}{15mm}%
}
\chapterstyle{ppgccufmgsf}
%    \end{macrocode}

% Aqui ?definida a formata?o da identifica?o (\emph{caption}) de figuras,
% tabelas e outros elementos flutuantes (algoritmos, etc.).
%
%    \begin{macrocode}
\captionnamefont{\bfseries\small}
\captiondelim{. }
\captiontitlefont{\small}
\newlength{\ppgcc@captionwidth}
\setlength{\ppgcc@captionwidth}{\textwidth}
\addtolength{\ppgcc@captionwidth}{-2cm}
\captionwidth{\ppgcc@captionwidth}
\changecaptionwidth
%    \end{macrocode}

% A seguir ser?reprogramado o comando que apresenta os n?meros de p?ina de
% cap?ulos no Sum?io. Esta reprograma?o evita que qualquer indica?o
% num?ica invada a margem direita do corpo do texto.
%
%    \begin{macrocode}
\renewcommand*{\cftchapterformatpnum}[1]{%
  \hb@xt@\@pnumwidth{\hss\cftchapterpagefont #1}}
%    \end{macrocode}


%
% \subsection*{Rotina de p?-processamento}
%
% \DescribeMacro{\AtBeginDocument}%
% Finalmente, a rotina atrelada ao comando |\AtBeginDocument|: aqui s?
% chamados os comandos para configurar a linguagem e o espa?mento padr? entre
% linhas. O pacote \pkg{graphicx} tamb? ?carregado automaticamente, embora
% seja melhor que o usu?io inclua manualmente esse pacote: dessa maneira, o
% usu?io poder?carreg?lo com as op?es que desejar.
%
%    \begin{macrocode}
\AtBeginDocument{%
  \ppgcc@selectlanguage\ppgcc@defspacing
  \@ifpackageloaded{graphicx}{}{\RequirePackage{graphicx}}%
  \@ifpackageloaded{hyperref}{\RequirePackage{memhfixc}}{}%
}
%    \end{macrocode}
%
% \PrintIndex
%
% \Finale
