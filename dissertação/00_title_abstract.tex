
A aplicação de sistemas autônomos pode ser observada em diversos cenários, como em atividades de vigilância, busca e resgate, rastreamento e mapeamento, além destas, o transporte e manipulação de objetos é um tipo de missão que tem ganhado atenção por parte de pesquisadores e grandes empresas, principalmente por acrescentar à atividade características como precisão, agilidade e controle, além de poder garantir melhoras de eficiência geral do sistema.
Neste trabalho é demonstrado um conjunto de técnicas utilizadas para a coordenação de uma equipe de agente robóticos heterogêneos, tendo como missão o transporte de objetos utilizando suas capacidades físicas e computacionais.
O processo de transporte e manipulação é tratado desenvolvido em três etapas: (i) planejamento de caminhos para os objetos, na qual um plano de movimentação do objeto é criado baseado em métricas de uso energético e custo de tempo; (ii) planejamento e alocação de tarefas entre os agentes disponíveis para o transporte, realizando a distribuição de tarefas de modo a minimizar a utilidade total do sistema; (iii) coordenação e execução, na qual os agentes realizam o transporte de forma sincronizada.
Foram executados experimentos quantitativos, demonstrando a efetividade do método avaliado em diversos cenários distintos, além de testes simulados e reais, que ilustram a manipulação dos objetos por agentes robóticos propriamente ditos.
Desta maneira, é demonstrado uma metodologia capaz de executar de forma completa a missão de transporte, aprimorando-a através do uso de agentes inteligentes.




%

The application of autonomous systems can be observed in various scenarios, such as surveillance, search and rescue, tracking and mapping, apart from these, transportation and manipulation of objects is a kind of mission that has gained attention from researchers and large companies, mainly by adding to the activity features like precision, speed and control, in addition to ensuring overall efficiency improvements to the system.
This work demonstrated a set of techniques used to coordinate a team of heterogeneous robotic agents, whose mission is to transport objects using their physical and computational skills.
The process of transport and manipulate is handled in three stages: (i) object path planning, in which a plan is created based on metrics of energy use and cost of time; (ii) planning and task allocation between the available agents for transportation, making the distribution of tasks in order to minimize the total utility of the system; (iii) coordination and execution, in which agents perform transport synchronously.
Were performed quantitative experiments, demonstrating the effectiveness of the method valued in several different scenarios, as well as simulated and real tests, illustrating the manipulation of objects by robotic agents themselves.
Thus, it is shown a methodology capable of performing a full transport mission, improving it through the use of intelligent agents.


Transporte Cooperativo, Manipulação de Objetos, Alocação de Tarefas





% Através da aplicação da técnica, foi possível realizar o transporte de um conjunto de objetos
