% Outras notas

# Aplicações de multiplos robôs

- Vigilância
- Reconhecimento
- Busca e Resgate
- Mapeamento
- Detecção e Rastramento de alvos
- Transporte de objetos
- Construção de estruturas

# Desafios

- Coordenação
- Falha na comunicação
- Incerteza da leitura de sensores

# Vantagens

- Eficiência
- Tolerância a falhas

#######################

Diagrama

# Path Planning
	- Occupancy Grid
	- Frontier Expansion
	- Probabilistic Road Map
# Cooperative Control
	- Reative
	- Centralized
	- Decentralized
	- Biological Inspired Decentralized
	- Hierarchical Architecture
	- Alpha Beta Coordination (Switched)
# Mapping
	- Semantic Map
	- Discrete Map
	- Topological Map

# Simultaneous task and motion planning (STAMP)
# Simultaneous task allocation and path planning
# Simultaneous Task Allocation and Motion Coordination (STAMC)

## Trabalhos Relacionados

Coordenação de Multi-Robôs
	Planejamento de Caminhos
		Grade de Ocupação					GO
		Expansão de Fronteira			EF
		Campos Potenciais					CP
		Grafo de Visibilidade			GVi
		Grafo de Voronoi					GVo
	Controle Cooperativo
		Comportamental - Behavior	Bh
		Reativo										Re
		Centralizado							Ce
		Descentralizado 					De
		Biologicamente Inspirado	BI
		Arquitetura Hierarquica		Hi
		AlphaBeta									AB
	Mapeamento
		Mapa Semântico						MS
		Mapa Topológico						MT
		Discreto ( matriz )				Di
	Alocação de Tarefas
		Leilão										Le

Coordenação de Multi-Robôs
	Planejamento de Caminhos
		Campos Potenciais
		Baseado em Grafo
			Grafo de Visibilidade
			Grafo de Voronoi
	Controle Cooperativo
		Comportamental
		Baseado em Autonomia de Desição
			Centralizado
			Descentralizado
		Biologicamente Inspirado
		AlphaBeta
	Mapeamento
		Baseado em Grafo
			Mapa Semântico
			Mapa Topológico
		Baseado em Matriz
			Mapa Discreto
			Matriz de Ocupação

#####################

Considerações

# Estado final do objeto será assinalado considerando somente sua posição.

# Os agentes serão capazes de realizar o transporte com no máximo 2 agentes o realizando ao mesmo tempo.

# A localização de todos as partes envolvidas no transporte serão conhecidas:
	- Agentes
	- Objetos
	- Obstáculos

# Não existirão outros agentes no ambiente se não aqueles que estão participando do transporte.

# Os objetos dispostos no ambientes estarão fixos e não se moverão durante o transporte.

##################################

Arquitetura do Sistema

Classes Constituintes

Controlador do Agente Terrestre
	- Controlador Local
	- Controlador Global

Planejador Nautilus
	- A*
	- PRM
		- Uniform Anchor Distribution
		- A* on graph
	- RRT

Controlador do Planejamento do Objeto
	- Planejamento do Trajeto do Objeto
	- Planejamento dos pontos de movimentação do robo

Alocador de Tarefas
	- Função de Alocação
		- Distancia entre o agente e o ponto de movimentação
		- Tipo de agente
		- Tipo de transporte

Controlador da Simulação
	- Deploy dos Agentes
	- Inicialização do Mundo
		- Obstaculos
		- Objeto

Sistema de Localização
	- Agentes
	- Objeto
	- Obstaculos



% Trabalhos Relacionados

\cite{Parra-Gonzalez2011}

Muito parecido, pois também foca o planejamento do objeto e procura diminuir a quantidade de vezes que os agentes mudam de posição para empurrar o objeto


\cite{Hart1968} A* Algoritm reference


% Ideias para Experimentos Quantitativos

- Usar varios cenarios diferentes, com diferentes tamanhos de áreas livres para movimentação (algo como uma percentagem livre)
- Usar tamanhos diferentes de células
-

Avaliar:

- Numero de reconfigurações dos robos, ou sejam, quantas vezes mudam de lado para empurar o objeto
- O tempo necessário para achar a solução, custo computacional em tempo para execução do algoritmo.
- Custo do transporte utilizando os diferentes metodos, ou quantidade de robos
-
